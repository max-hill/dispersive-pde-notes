\documentclass{article}
\usepackage[utf8]{inputenc}

% Math Packages
\usepackage{amsmath, mathtools}
\usepackage{amssymb}
\usepackage{amsthm}
\usepackage{amsfonts}
\usepackage{bbm}
\usepackage{breqn}
\usepackage[margin=1in]{geometry}
\usepackage{graphicx}
\usepackage{tikz}
\usepackage{forest}
\usepackage{tikz-qtree}
\graphicspath{ {./images/} }
\usepackage{hyperref}
\usepackage[shortlabels]{enumitem}
\usetikzlibrary{arrows,matrix,positioning}
\usepackage[ruled,vlined]{algorithm2e}

% References
\usepackage[capitalize]{cleveref}

% Colorful Notes
\usepackage{color}
\definecolor{Red}{rgb}{1,0,0}
\definecolor{Blue}{rgb}{0,0,1}
\definecolor{Purple}{rgb}{.5,0,.5}
\def\red{\color{Red}}
\def\blue{\color{Blue}}
\def\gray{\color{gray}}
\def\purple{\color{Purple}}

\newcommand{\rnote}[1]{{\red [#1] }} % \rnote{foo} then 'foo' is red
\newcommand{\bnote}[1]{{\blue #1}} % \bnote{foo} then 'foo' is blue
\newcommand{\gnote}[1]{{\gray [#1]}} % gray note  -- for comments
\newcommand{\pnote}[1]{{\purple [#1]}} % green note  -- for comments

% Math Environments
\newtheorem{theorem}{Theorem}
\newtheorem{definition}{Definition}
\newtheorem{assumption}{Assumption}
\newtheorem{lemma}{Lemma}
\newtheorem{remark}{Remark}
\newtheorem{prop}{Proposition}
\newtheorem{corollary}{Corollary}
\newtheorem{example}{Example} 
\newtheorem{question}{Question}
\newtheorem{claim}{Claim} 
\newtheorem{conjecture}{Conjecture} 

% Custom Math Commands
\newcommand{\vt}{\vskip 5mm} % vertical space
\newcommand{\fl}{\noindent\textbf} % first line
\newcommand{\Fl}{\vt\noindent\textbf} % first line with space above
\def\R{\mathbb{R}} % Real numbers
\def\Z{\mathbb{Z}} % Integers
\def\E{\mathbb{E}} % Expectation
\def\P{\mathbb{P}} % Probability
\def\Q{\mathbb{Q}} % Q probability
\newcommand{\indep}{\perp \!\!\! \perp}  %independence symbol
\newcommand{\ifelse}[4]{\left\{ \begin{array}{l@{\quad:\quad}l}#1&#3\\#2&#4\end{array}\right.}
\newcommand{\curly}[2]{\left\{ \begin{array}{l@{\quad}l}#1\\#2\end{array}\right.}


\usepackage[utf8]{inputenc}
\usepackage{hyperref}
\usepackage{amsmath}
\usepackage[shortlabels]{enumitem}
\newtheorem{problem}{Problem}

% Concise Vector Macro: write bracket vectors of arbitrary length
% Usage: \columnvector{x,y,1} or \columnvector{2,x^2}
\usepackage{stackengine}
\newcommand\columnvector[1]{\setstackEOL{,}\bracketVectorstack{#1}}


\title{Mihaela Ifrim's Lecture Notes on Dispersive PDE}
\date{Fall 2022}


\begin{document}
\section{Lecture 1: 2022-09-08}
Not typed up yet.

\section{Lecture 2: 2022-09-13}
Recall the dispersion relation $a(\xi)$. If $a''(\xi) \neq 0$ then the PDE is
said to be \textbf{fully dispersive}. If $\xi\in \R^{d}$ then $\nabla^{2}a(\xi)$
is the hessian. The appropriate thing to do in that case is to diagonalize...
we'll get to that.

\begin{example}[Wave Equation]
  Suppose we have $\square u = 0$ with $d=1$. Then $a(\xi) = |\xi|$ so that
  $a''(\xi)=0$ which is not dispersive. But if $d>1$ then since
  $\alpha(\xi)=|\xi|$. So that
  $a'(\xi)= -\nabla a(\xi)= \frac{\xi}{\left| \xi \right|}$. Then the hessian is
  \begin{equation*}
    \nabla^{2}_{\xi} a(\xi) 
    = \frac{1}{\left| \xi \right|}\left[ I_{n}-\frac{\xi}{|\xi|}\otimes \frac{\xi}{|\xi|} \right] 
  \end{equation*}
  where $I_{n}$ is the $n\times n$ identity matrix. The thing in brackets is the
  orthogonal projection onto the plane perpendicular to $\xi/|\xi|$ (i.e.
  tangent to the sphere).

  For example, consider $d=2$. Then we can diagonalize the Hessian matrix
  $\nabla^{2}_{\xi}a$.  You get $\lambda=0$ and $\lambda= \frac{1}{|\xi|}$ as
  eigenvalues and the diagonalized matrix is
  \begin{equation*}
    \begin{pmatrix}
      0&0\\
      0&\frac{1}{|\xi|}
    \end{pmatrix}
  \end{equation*}
  The interpretation is that since one of the eigenvalues is zero, so you don't
  have full dispersion. It's like degenerate dispersion. In particular, we don't
  have disperion in the radian direction. Try playing with this in the 2D. 
\end{example}
\begin{remark}
  The wave equation in $d>1$ has dispersion (degenerate). Also, last lecture had
  something false in it. The false claim was that finite speed of propagation is
  impied by bounded group velocity. (Recall that the \textbf{group velocity} is
  the negative of the derivative of the dispersion relation: $-\nabla a(\xi)$.) 
\end{remark}
\begin{example}
  Suppose $u:\R\times\R^{3}\to \R$ satisfying
  \begin{equation}\label{eq:1}
    \curly{i\partial_{t}u +A(D)u =0}{u_{0}=u(0,\lambda)}
  \end{equation}
  We claim that this equation with $A(D)=|D|$ does not have finite speed of
  propagation.

  Finite speed propagation is if $u_{0}$ is supported on $B(a,R)$ then $t>0$,
  $u(x,t)$ is going to be supported by $B(a,R+ct)$. The infimum of all such $c$'s
  satifying the above condition is called \textbf{the finite speed of
    propagation.}

  Recall that we can take the Fourier transform of \cref{eq:1} and solve for the
  fundamental solution:
  \begin{equation*}
    e^{it|D|} = \cos(t|D|)+ i \sin(t|D|)
  \end{equation*}
  here $D$ is the derivative with radial.
  \begin{equation*}
    |D| \frac{\sin(t|D|)}{|D|}
  \end{equation*}
  Recall the fundamental solution to the wave equation is
  $K_{\square}= c \frac{1}{t}\delta$. Give this a little bit of thought. Do this
  example to bush up. This example shows why the finite group velocity does not
  imply finite speed of propagation. This example looks like the example from
  last time that had a star with it. The operatior $|D|$ is the operator with
  symbol $|\xi|$. In one dimension, this is the Hilbert transform.
\end{example}

\Fl{Question:} What are the long time dynamics for linear dispersive waves?


Scalar case. Suppose we have
\begin{equation*}
  \curly{i\partial_{t}u +A(D)u=0}{u_{0}(x)=u(0,x)\in \mathcal{S}}
\end{equation*}
where $\mathcal{S}$ denotes the Schwarz space. What happens with this wave as
$t \to \infty$?

Then the solution is
\begin{equation*}
  \hat{u}(t,\xi) = \hat{u}_{0}e^{it a(\xi)}
\end{equation*}
where $a(\xi)$  is the symbol of the operator $A$. Remember, dispersivity is
that every frequency moves in its own direction and with its own velocity. We
want to quantify this. My solution:
\begin{equation}\label{eq:3}
  \hat{u}(t,\xi) = \hat{u}_{0}(\xi)e^{ita(\xi)}
\end{equation}
now let's go back and write it on the spatial side:
\begin{equation*}
  u(t,x) = \int e^{ix\cdot\xi}e^{ita(\xi)}\hat{u}_{0}(\xi)d\xi
\end{equation*}
where $\hat{u}_{0}(\xi)$ is a Schwarz function (since $u_{0}\in \mathcal{S}$ and
the Fourier Transform maps Schwarz functions to Schwarz functions). Therefore
\begin{equation*}
  u(t,x) = \int e^{i(x\cdot\xi+ta(\xi))}\hat{u}_{0}(\xi)d\xi
\end{equation*}
Heuristically, we make the assumption that the waves (each localized a different
frequency) travel and go out in a linear fasion (at least when time is big).
When $t$ is very large, we write
\begin{equation*}
  u(t,x) = \int e^{it \left[ \frac{x}{t}\xi+a(\xi) \right] }\hat{u}_{0}(\xi)d\xi
\end{equation*}
and we think of $v:=x/t$ as velocity. Thus
\begin{equation*}
  u= u(t,v) = \int e^{it \left[ v\xi+a(\xi) \right] }\hat{u}_{0}(\xi)d\xi
\end{equation*}
and this is an oscillatory integral. This is oscillate rapidly when $t$ is
large. One nice thing about it is that it has a complex phase, which is what
makes it oscillate. We are interested in the long-time dynamics. That will
result in cancellations, which can be seen when integrating by parts. There is a
complication when doing integration by parts however because we don't know that
the quantity 
\begin{equation*}
  \frac{\partial }{\partial \xi} \left[ itv\xi +a(\xi) \right] 
\end{equation*}
is nonzero and integration by parts would require it to be on the denominator.
So we need to introduce something called stationary/nonstationary phase. In
$d=1$, this is from Stein. The main understanding is the following:

Let
\begin{equation*}
  I = \int e^{i\lambda\phi(\xi)}a(\xi)d\xi 
\end{equation*}
First, suppose that $\phi(\xi) \neq 0$. This is called the nonstationary phase
argument. Take
\begin{equation*}
  I 
  = \int \partial_{\xi} \left( e^{i\lambda \phi} \right)\frac{1}{i\lambda \phi'(\xi)} a(\xi)d\xi  
\end{equation*}
Integrating by parts $N$ times will give a $\lambda^{-N}$ in front (e.g. if $a$
is a polynomial then $N$ would be the degree of $a$). Therefore as
$t\to \infty$,
\begin{equation*}
  I = O(\lambda^{-N})
\end{equation*}
which decays very quickly. Therefore the solution of \cref{eq:3}, understood as
a function $u=u(t,v)$ disperses very fast---it decays rapidly.

On the other hand, suppose that $\phi'(\xi) = 0$. For example, if $\phi(\xi)$ is
a constant. But that doesn't makeif $\xi_{0}$ is a critical point sense for our
problem. So let's correct for that by requiring that
\begin{equation*}
  \phi''(\xi) \neq 0.
\end{equation*}
Thus any zeros are when we have critical points of $\phi$. Around those points,
we cannot integrate by parts. But what can we do? Another heuristic: if
$\xi_{0}$ is a critical point, assume that when $\xi\approx \xi_{0}$, we can
Taylor expand:
\begin{align*}
  \phi(\xi) 
  &= \phi(\xi_{0}) + (\xi-\xi_{0})\phi(\xi_{0}) + \frac{1}{2}(\xi-\xi_{0})^{2}\phi''(\xi) \\
  &= \phi(\xi_{0}) + \frac{1}{2}(\xi-\xi_{0})^{2}\phi''(\xi).
\end{align*}
so that
\begin{align*}
  I 
  &\approx \int e^{i\lambda \left[ \phi(\xi_{0})+\frac{1}{2}(\xi-\xi_{0})^{2}\phi''(\xi_{0}) \right] } a(\xi_{0})d\xi \\
  &= e^{i\lambda \phi(\xi_{0})}a(\xi_{0}) \int e^{\frac{1}{2}i\lambda (\xi-\xi_{0})^{2}\phi''(\xi_{0})}d\xi
\end{align*}
and the integral is a complex Gaussian which can be computed with the rule
\begin{equation*}
  \int_{0}^{\infty}e^{i\alpha x^{2}}dx 
  = e^{i\pi \text{sign}(|\alpha|)/4} \sqrt{\frac{\pi}{4\alpha}}
  , \quad \alpha  \neq 0.
\end{equation*}
with $\alpha = \frac{1}{2}\lambda \phi''(\xi_{0})$. So we get
\begin{equation*}
  I 
  \approx e^{i \lambda \phi(\xi_{0})} a(\xi_{0}) \frac{1}{\sqrt{\lambda}} \cdot \frac{1}{\sqrt{\phi''(\xi_{0})}}
\end{equation*}
multiplied by some other thing that aren't important. Taking $\phi = v\xi +
a(\xi)$  and $\lambda= t$ gives
\begin{equation*}
  u(t,vt) 
  \approx e^{it \left[ v_{\xi}\xi_{0}+ a(\xi_{0}) \right] }\frac{1}{\sqrt{t}}\hat{u}_{0}(\xi_{v})
\end{equation*}
with some other things.
\end{document}
