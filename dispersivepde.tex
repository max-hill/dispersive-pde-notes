\documentclass{article}
\usepackage[utf8]{inputenc}

% Math Packages
\usepackage{amsmath, mathtools}
\usepackage{amssymb}
\usepackage{amsthm}
\usepackage{amsfonts}
\usepackage{bbm}
\usepackage{breqn}
\usepackage[margin=1in]{geometry}
\usepackage{graphicx}
\usepackage{tikz}
\usepackage{forest}
\usepackage{tikz-qtree}
\graphicspath{ {./images/} }
\usepackage{hyperref}
\usepackage[shortlabels]{enumitem}
\usetikzlibrary{arrows,matrix,positioning}
\usepackage[ruled,vlined]{algorithm2e}

% References
\usepackage[capitalize]{cleveref}

% Colorful Notes
\usepackage{color}
\definecolor{Red}{rgb}{1,0,0}
\definecolor{Blue}{rgb}{0,0,1}
\definecolor{Purple}{rgb}{.5,0,.5}
\def\red{\color{Red}}
\def\blue{\color{Blue}}
\def\gray{\color{gray}}
\def\purple{\color{Purple}}

\newcommand{\rnote}[1]{{\red [#1] }} % \rnote{foo} then 'foo' is red
\newcommand{\bnote}[1]{{\blue #1}} % \bnote{foo} then 'foo' is blue
\newcommand{\gnote}[1]{{\gray [#1]}} % gray note  -- for comments
\newcommand{\pnote}[1]{{\purple [#1]}} % green note  -- for comments

% Math Environments
\newtheorem{theorem}{Theorem}
\newtheorem{definition}{Definition}
\newtheorem{assumption}{Assumption}
\newtheorem{lemma}{Lemma}
\newtheorem{remark}{Remark}
\newtheorem{prop}{Proposition}
\newtheorem{corollary}{Corollary}
\newtheorem{example}{Example} 
\newtheorem{question}{Question}
\newtheorem{claim}{Claim} 
\newtheorem{conjecture}{Conjecture} 

% Custom Math Commands
\newcommand{\vt}{\vskip 5mm} % vertical space
\newcommand{\fl}{\noindent\textbf} % first line
\newcommand{\Fl}{\vt\noindent\textbf} % first line with space above
\def\R{\mathbb{R}} % Real numbers
\def\Z{\mathbb{Z}} % Integers
\def\E{\mathbb{E}} % Expectation
\def\P{\mathbb{P}} % Probability
\def\Q{\mathbb{Q}} % Q probability
\newcommand{\indep}{\perp \!\!\! \perp}  %independence symbol
\newcommand{\ifelse}[4]{\left\{ \begin{array}{l@{\quad:\quad}l}#1&#3\\#2&#4\end{array}\right.}
\newcommand{\curly}[2]{\left\{ \begin{array}{l@{\quad}l}#1\\#2\end{array}\right.}


\usepackage[utf8]{inputenc}
\usepackage{hyperref}
\usepackage{amsmath}
\usepackage[shortlabels]{enumitem}
\newtheorem{problem}{Problem}

% Concise Vector Macro: write bracket vectors of arbitrary length
% Usage: \columnvector{x,y,1} or \columnvector{2,x^2}
\usepackage{stackengine}
\newcommand\columnvector[1]{\setstackEOL{,}\bracketVectorstack{#1}}


\title{Lecture Notes for Mihaela Ifrim's Course on Dispersive PDE}
\date{Fall 2022}


\begin{document}
\title{Statistics }
\author{Max Hill}
\maketitle
\tableofcontents
\newpage

\section{Lecture 1: 2022-09-08}
\subsection{Introduction to Dispersive PDEs}
\begin{definition}[Dispersive PDE]
  Informally, a PDE is characterized as \textbf{dispersive} if, when the the
  boundary conditions are dropped, its wave solutions are going to spread out in
  space as time evovles. Like a rock thrown into water.
\end{definition}

For now we will focus on \textit{linear} dispersive PDEs. Associated with a
dispersive PDE:
\begin{enumerate}
  \item Dispersive Estimates
  \item On the Fourier side, different frequencies at different speedss in
  different directions.
\end{enumerate}
We first consider the simplest example.

\begin{example}[Linear dispersive PDE with constant coefficients] \label{ex:linear-dispersive-pde-with-constant-coefficients}
  Suppose $u(t,x):\R\times\R^{d}\to V \in \R^{d}$ such that
  \begin{equation}
    \label{eq:linear-constant-coefficient-pde}
    \curly{i\partial_{t}u(t,x)=Lu(t,x)}{u(0,x)=u_{0}(x)}
  \end{equation}
  where $L$ is a skew-adjoint constant coeffieicent differential operator of
  order $k$. In symbols, there exists constants $\left\{c_{\alpha}: \alpha\in
    \Z_{\geq 0}^{d}\right\}$ such that
  \begin{equation*}
    Lu(t,\cdot) = \sum_{|\alpha| \leq k}c_{\alpha} \partial_{x}^{\alpha}u(t,\cdot)
  \end{equation*}
  where $|\alpha| := \alpha_{1}+\ldots+\alpha_{d}$ and for $x\in \R^{d}$, the
  symbol $\partial_{x}^{\alpha}$ denotes the operator defined by
  \begin{equation}
    \label{eq:2} 
    \partial_{x}^{\alpha}u := \prod_{i=1}^{d} \partial_{x_{i}}^{\alpha_{i}}u
  \end{equation}
  So, for example, if $d=2$, $c_{(0,1)}=2$, $c_{(2,5)}=-3$, and
  $c_{\alpha}=0$ for all other choices of $\alpha$, then $L$ would be an order 7
  differential operator taking the following form:
  \begin{align*} 
    Lu(t,\cdot)
    &= 2 \partial_{x}^{(0,1)}u(t,\cdot)-3 \partial_{x}^{(2,5)}u(t,\cdot)\\
    &= 2\partial_{x_{1}}^{0}u(t,\cdot)\partial_{x_{2}}^{1}u(t,\cdot) -3\partial_{x_{1}}^{2}u(t,\cdot)\partial_{x_{2}}^{5}u(t,\cdot)
  \end{align*}
  or equivalently,
  \begin{equation*}
    Lu(t,x_{1},x_{2}) = 2u(t,x_{1},x_{2})u_{x_{2}}(t,x_{1},x_{2}) -3u_{x_{1}x_{1}}(t,x_{1},x_{2})u_{x_{2}x_{2}x_{2}x_{2}x_{2}}(t,x_{1},x_{2})
  \end{equation*}
  Writing $x,y$ in place of $x_{1},x_{2}$, we get the nicer-looking formulation:
  \begin{equation*}
    Lu(t,x) = 2u(t,x,y)u_{y}(t,x,y)-3u_{xx}(t,x,y)u_{yyyyy}(t,x,y).
  \end{equation*}
  Since the operator defined in \cref{eq:2} does not compose different partial
  differentiation operators together, $Lu$ does not involve any mixed partial derivatives of $u$ (e.g.
  you'll never see terms like $u_{xy}$ or $u_{xxy}$).
\end{example}

\begin{remark}
  The operator $L$ from
  \cref{ex:linear-dispersive-pde-with-constant-coefficients} is defined
  classically (i.e. pointwise) only if $u\in C^{k}(\R\times\R^{d})$, that is
  only if $u$ is $k$-times continuously differentiable. But of course we may
  extend $L$ in the usual manner so that it is defined in a distributional
  sense.)
\end{remark}

\subsection{Dispersion Relation}
\begin{definition}[Dispersion Relation and Frequency Operator]
  \label{def:dispersion-relation-frequency-operator}
  We can write $L$ in the form
  \begin{equation*}
    L = ia(D)
  \end{equation*}
  where $a$  is some function, called the \textbf{dispersion relation}, and $D$
  is the \textbf{frequency operator} defined as
  \begin{equation*}
    D = \frac{1}{i}\nabla := \left( \frac{1}{i}\partial_{x_{1}},\ldots,\frac{1}{i}\partial_{x_{d}} \right) 
  \end{equation*}
\end{definition}
\begin{remark}[Polynomial Dispersion Relations]
  \label{rmk:polynomial-dispersion-relation}
  It turns out that the form of the dispersion relation is important. When the
  operator $L$  is of the form
  \begin{equation*}
    Lu= \partial_{t}u
  \end{equation*}
  (or something), then the dispersion relation
  takes the form
  \begin{equation}
    \label{eq:polynomial-dispersion-relation}
    a(\xi_{1},\ldots,\xi_{d})= \sum_{|\alpha| \leq k}i^{|\alpha|-1}c_{\alpha}\xi_{1}^{\alpha_{1}}\cdots \xi_{d}^{\alpha_{d}}
  \end{equation}
\end{remark}
A large number of PDEs are governed by dispersion relations of the form given
in \cref{eq:polynomial-dispersion-relation}, as we show in the next example.

\begin{example}[Examples for \cref{rmk:polynomial-dispersion-relation}]
  Here we list some examples of PDEs whose dispersion relations take the form
  shown in \cref{eq:polynomial-dispersion-relation}.
  \begin{itemize}
    \item \textbf{A degenerate (i.e. nondispersive) example.} Suppose
    \cref{eq:linear-constant-coefficient-pde} takes the form
    \begin{equation*}
      \curly{\partial_{t}u(t,x) = iwu(t,x)}{u(0,x)=u_{0}(x)}
    \end{equation*}
    where $u:\R\times\R\to\R$ and $w\in \R$. In this case the dispersion
    relation is
    \begin{equation*}
      a(\xi) = w.
    \end{equation*}
    The solution to this system is $u(t,x)= u_{0}e^{itw}.$
    \item \textbf{Another degenerate equation.} Fix $\nu\in \R^{d}$ and
    consider the
    \begin{equation*}
      \curly{\partial_{t}u(t,x) = -\nu\cdot \nabla_{x}u(t,x)}{u_{0}(0,x)=u_{0}(x)}
    \end{equation*}
    An example solution to this is
    \begin{equation*}
      u(t,x) = u_{0}(x-\nu t)
    \end{equation*}
    something something transport equation. 

    \item \textbf{The Airy Equation.} Suppose $d=1$, $u=u(t,x):\R\times\R \to \R$ such
    that
    \begin{equation*}
      \partial_{t}u +\partial_{xxx}u=0.
    \end{equation*}
    In this case, the dispersion relation is
    \begin{equation*}
      a(\xi) = \xi^{3},\quad (\xi\in \R).
    \end{equation*}

    \item \textbf{Schrodinger (?) equation.} Suppose
    \begin{equation*}
      i\partial_{t}u +\Delta u = 0.
    \end{equation*}
    In this case, $L=-\Delta$ and $a(\xi)=|\xi|^{2}$. 
    \item \textbf{Wave Equation.} Suppose
    \begin{equation}\label{eq:wave-equation}
      -u_{tt}+\Delta u = 0.
    \end{equation}
    in any dimnensions. This is dispersive.
    \item \textbf{Klein-Gordon Equation.} Suppose
    \begin{equation*}
      -u_{tt} + \Delta u - u = 0.
    \end{equation*}
    This is dispersive.
  \end{itemize}
\end{example}

It is not always the case that dispersion relations are polynomials of the form
given in \cref{eq:polynomial-dispersion-relation}. Indedd, for deep water
gravity waves, $a(\xi)= |\xi|^{1/2}$. Also for capillary waves, something
something. Also the BO equation
\begin{equation*}
  u_{t}+ H \partial_{x}^{2}u = 0
\end{equation*}
has dispersion relation $a(\xi)=\xi|\xi|$.
\subsection{Group Velocity}
\begin{definition}[Wave-Plane Solution]
  \label{def:wave-plane-solution}
  A \textbf{wave-plane solution} is the name for functions of the form
  \begin{equation}
    \label{eq:wave-plane-solution}
    u(t,v) = e^{i(kx-wt)}.
  \end{equation}
\end{definition}
\begin{definition}[Wave Number and Angular Frequency]
  \label{def:wave-number-angular-frequency}
  The parameter $k$ in \cref{eq:wave-plane-solution} is called the \textbf{wave
    number}. The parameter $w$ is called the \textbf{angular frequency}.
\end{definition}
\begin{example}[Wave-plane solutions for the wave equation]
  If we plug the function $u$ from \cref{eq:wave-plane-solution} into
  the wave equation shown in \cref{eq:wave-equation}, we obtain
  \begin{equation*}
    -(-iw)^{2}e^{(i(kx-wt))} +(ik)^{2}e^{(i(kx-wt))} =0
  \end{equation*}
  which implies that $w^{2}=k^{2}$, and hence that $w(k) = \pm |k|.$
\end{example}
\begin{definition}[Group Velocity]
  The derivative of the dispersion relation with respect to the dwave number is
  called the \textbf{group velocity}.
\end{definition}
\begin{definition}[Dispersive Equation]
  A PDE is said to be dispersive if $a''(\xi) \neq 0$. 
\end{definition}

\subsection{Symmetries for Linear Dispersive PDEs}
\begin{enumerate}
  \item All are invariant under time and space translations. That is, suppose
  $\tau$ is the translation operator defined by
  \begin{equation*}
    \tau u(t,x) = u(t-t_{0},x-x_{0})
  \end{equation*}
  for some fixed $t_{0}$ and $x_{0}$. Then $\tau u$ is a solution provided that
  $u$ is a solution. 
  \item Scaling symmetries. For any $\lambda>0$,  the function
  \begin{equation*}
    u(t/\lambda^{k},x/\lambda)
  \end{equation*}
  is a solution whenever $u(t,x)$ is a solution.
  \item I think there were more, but my hand got tired and I stopped taking notes.
\end{enumerate}

\section{Lecture 2: 2022-09-13}
\subsection{Full Dispersion}
Recall the dispersion relation $a(\xi)$. If $a''(\xi) \neq 0$ then the PDE is
said to be \textbf{fully dispersive}. If $\xi\in \R^{d}$ then $\nabla^{2}a(\xi)$
is the hessian. The appropriate thing to do in that case is to diagonalize...
we'll get to that.

\begin{example}[Wave Equation]
  Suppose we have $\square u = 0$ with $d=1$. Then $a(\xi) = |\xi|$ so that
  $a''(\xi)=0$ which is not dispersive. But if $d>1$ then since
  $\alpha(\xi)=|\xi|$. So that
  $a'(\xi)= -\nabla a(\xi)= \frac{\xi}{\left| \xi \right|}$. Then the hessian is
  \begin{equation*}
    \nabla^{2}_{\xi} a(\xi) 
    = \frac{1}{\left| \xi \right|}\left[ I_{n}-\frac{\xi}{|\xi|}\otimes \frac{\xi}{|\xi|} \right] 
  \end{equation*}
  where $I_{n}$ is the $n\times n$ identity matrix. The thing in brackets is the
  orthogonal projection onto the plane perpendicular to $\xi/|\xi|$ (i.e.
  tangent to the sphere).

  For example, consider $d=2$. Then we can diagonalize the Hessian matrix
  $\nabla^{2}_{\xi}a$.  You get $\lambda=0$ and $\lambda= \frac{1}{|\xi|}$ as
  eigenvalues and the diagonalized matrix is
  \begin{equation*}
    \begin{pmatrix}
      0&0\\
      0&\frac{1}{|\xi|}
    \end{pmatrix}
  \end{equation*}
  The interpretation is that since one of the eigenvalues is zero, so you don't
  have full dispersion. It's like degenerate dispersion. In particular, we don't
  have disperion in the radian direction. Try playing with this in the 2D. 
\end{example}
\begin{remark}
  The wave equation in $d>1$ has dispersion (degenerate). Also, last lecture had
  something false in it. The false claim was that finite speed of propagation is
  impied by bounded group velocity. (Recall that the \textbf{group velocity} is
  the negative of the derivative of the dispersion relation: $-\nabla a(\xi)$.) 
\end{remark}
\begin{example}
  Suppose $u:\R\times\R^{3}\to \R$ satisfying
  \begin{equation}\label{eq:1}
    \curly{i\partial_{t}u +A(D)u =0}{u_{0}=u(0,\lambda)}
  \end{equation}
  We claim that this equation with $A(D)=|D|$ does not have finite speed of
  propagation.

  Finite speed propagation is if $u_{0}$ is supported on $B(a,R)$ then $t>0$,
  $u(x,t)$ is going to be supported by $B(a,R+ct)$. The infimum of all such $c$'s
  satifying the above condition is called \textbf{the finite speed of
    propagation.}

  Recall that we can take the Fourier transform of \cref{eq:1} and solve for the
  fundamental solution:
  \begin{equation*}
    e^{it|D|} = \cos(t|D|)+ i \sin(t|D|)
  \end{equation*}
  here $D$ is the derivative with radial.
  \begin{equation*}
    |D| \frac{\sin(t|D|)}{|D|}
  \end{equation*}
  Recall the fundamental solution to the wave equation is
  $K_{\square}= c \frac{1}{t}\delta$. Give this a little bit of thought. Do this
  example to bush up. This example shows why the finite group velocity does not
  imply finite speed of propagation. This example looks like the example from
  last time that had a star with it. The operatior $|D|$ is the operator with
  symbol $|\xi|$. In one dimension, this is the Hilbert transform.
\end{example}


\subsection{Long-Time Dynamics for Linear Dispersive Waves}
\Fl{Question:} What are the long time dynamics for linear dispersive waves?


Scalar case. Suppose we have
\begin{equation*}
  \curly{i\partial_{t}u +A(D)u=0}{u_{0}(x)=u(0,x)\in \mathcal{S}}
\end{equation*}
where $\mathcal{S}$ denotes the Schwarz space. What happens with this wave as
$t \to \infty$?

Then the solution is
\begin{equation*}
  \hat{u}(t,\xi) = \hat{u}_{0}e^{it a(\xi)}
\end{equation*}
where $a(\xi)$  is the symbol of the operator $A$. Remember, dispersivity is
that every frequency moves in its own direction and with its own velocity. We
want to quantify this. My solution:
\begin{equation}\label{eq:3}
  \hat{u}(t,\xi) = \hat{u}_{0}(\xi)e^{ita(\xi)}
\end{equation}
now let's go back and write it on the spatial side:
\begin{equation*}
  u(t,x) = \int e^{ix\cdot\xi}e^{ita(\xi)}\hat{u}_{0}(\xi)d\xi
\end{equation*}
where $\hat{u}_{0}(\xi)$ is a Schwarz function (since $u_{0}\in \mathcal{S}$ and
the Fourier Transform maps Schwarz functions to Schwarz functions). Therefore
\begin{equation*}
  u(t,x) = \int e^{i(x\cdot\xi+ta(\xi))}\hat{u}_{0}(\xi)d\xi
\end{equation*}
Heuristically, we make the assumption that the waves (each localized a different
frequency) travel and go out in a linear fasion (at least when time is big).
When $t$ is very large, we write
\begin{equation*}
  u(t,x) = \int e^{it \left[ \frac{x}{t}\xi+a(\xi) \right] }\hat{u}_{0}(\xi)d\xi
\end{equation*}
and we think of $v:=x/t$ as velocity. Thus
\begin{equation*}
  u= u(t,v) = \int e^{it \left[ v\xi+a(\xi) \right] }\hat{u}_{0}(\xi)d\xi
\end{equation*}
and this is an oscillatory integral. This is oscillate rapidly when $t$ is
large. One nice thing about it is that it has a complex phase, which is what
makes it oscillate. We are interested in the long-time dynamics. That will
result in cancellations, which can be seen when integrating by parts. There is a
complication when doing integration by parts however because we don't know that
the quantity 
\begin{equation*}
  \frac{\partial }{\partial \xi} \left[ itv\xi +a(\xi) \right] 
\end{equation*}
is nonzero and integration by parts would require it to be on the denominator.
So we need to introduce something called stationary/nonstationary phase. In
$d=1$, this is from Stein. The main understanding is the following:

Let
\begin{equation*}
  I = \int e^{i\lambda\phi(\xi)}a(\xi)d\xi 
\end{equation*}
First, suppose that $\phi(\xi) \neq 0$. This is called the nonstationary phase
argument. Take
\begin{equation*}
  I 
  = \int \partial_{\xi} \left( e^{i\lambda \phi} \right)\frac{1}{i\lambda \phi'(\xi)} a(\xi)d\xi  
\end{equation*}
Integrating by parts $N$ times will give a $\lambda^{-N}$ in front (e.g. if $a$
is a polynomial then $N$ would be the degree of $a$). Therefore as
$t\to \infty$,
\begin{equation*}
  I = O(\lambda^{-N})
\end{equation*}
which decays very quickly. Therefore the solution of \cref{eq:3}, understood as
a function $u=u(t,v)$ disperses very fast---it decays rapidly.

On the other hand, suppose that $\phi'(\xi) = 0$. For example, if $\phi(\xi)$ is
a constant. But that doesn't makeif $\xi_{0}$ is a critical point sense for our
problem. So let's correct for that by requiring that
\begin{equation*}
  \phi''(\xi) \neq 0.
\end{equation*}
Thus any zeros are when we have critical points of $\phi$. Around those points,
we cannot integrate by parts. But what can we do? Another heuristic: if
$\xi_{0}$ is a critical point, assume that when $\xi\approx \xi_{0}$, we can
Taylor expand:
\begin{align*}
  \phi(\xi) 
  &= \phi(\xi_{0}) + (\xi-\xi_{0})\phi(\xi_{0}) + \frac{1}{2}(\xi-\xi_{0})^{2}\phi''(\xi) \\
  &= \phi(\xi_{0}) + \frac{1}{2}(\xi-\xi_{0})^{2}\phi''(\xi).
\end{align*}
so that
\begin{align*}
  I 
  &\approx \int e^{i\lambda \left[ \phi(\xi_{0})+\frac{1}{2}(\xi-\xi_{0})^{2}\phi''(\xi_{0}) \right] } a(\xi_{0})d\xi \\
  &= e^{i\lambda \phi(\xi_{0})}a(\xi_{0}) \int e^{\frac{1}{2}i\lambda (\xi-\xi_{0})^{2}\phi''(\xi_{0})}d\xi
\end{align*}
and the integral is a complex Gaussian which can be computed with the rule
\begin{equation*}
  \int_{0}^{\infty}e^{i\alpha x^{2}}dx 
  = e^{i\pi \text{sign}(|\alpha|)/4} \sqrt{\frac{\pi}{4\alpha}}
  , \quad \alpha  \neq 0.
\end{equation*}
with $\alpha = \frac{1}{2}\lambda \phi''(\xi_{0})$. So we get
\begin{equation*}
  I 
  \approx e^{i \lambda \phi(\xi_{0})} a(\xi_{0}) \frac{1}{\sqrt{\lambda}} \cdot \frac{1}{\sqrt{\phi''(\xi_{0})}}
\end{equation*}
multiplied by some other thing that aren't important. Taking $\phi = v\xi +
a(\xi)$  and $\lambda= t$ gives
\begin{equation*}
  u(t,vt) 
  \approx e^{it \left[ v_{\xi}\xi_{0}+ a(\xi_{0}) \right] }\frac{1}{\sqrt{t}}\hat{u}_{0}(\xi_{v})
\end{equation*}
with some other things.

\section{2022-09-15: Lecture Notes}

\subsection{Littlewood-Payley Decomposition}
Let $\phi$ be a smooth radial\footnote{A function $\phi$ is \textbf{radial} iff
  $\phi(x)=\phi(|x|)$ for all $x$.} function such that
\begin{equation*}
  \curly{\text{supp($\phi$ )}\subseteq \left\{\xi\in \R^{n}: 0 \leq |\xi| \leq
      2\right\}}{\phi\equiv 1 \text{ in }B(0,1)}
\end{equation*}
where $B(0,1)\subset \R^{n}$ is the unit ball centered at the origin. 
\end{document}
