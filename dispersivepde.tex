\documentclass{article}
\usepackage[utf8]{inputenc}

% Math Packages
\usepackage{amsmath, mathtools}
\usepackage{amssymb}
\usepackage{amsthm}
\usepackage{amsfonts}
\usepackage{bbm}
\usepackage{breqn}
\usepackage[margin=1in]{geometry}
\usepackage{graphicx}
\usepackage{tikz}
\usepackage{forest}
\usepackage{tikz-qtree}
\graphicspath{ {./images/} }
\usepackage{hyperref}
\usepackage[shortlabels]{enumitem}
\usetikzlibrary{arrows,matrix,positioning}
\usepackage[ruled,vlined]{algorithm2e}
% References
\usepackage[capitalize]{cleveref}

% This code makes widecheck for Fourier transform. Usage: \widecheck{blah blach}
\makeatletter
\DeclareRobustCommand\widecheck[1]{{\mathpalette\@widecheck{#1}}}
\def\@widecheck#1#2{%
  \setbox\z@\hbox{\m@th$#1#2$}%
  \setbox\tw@\hbox{\m@th$#1%
    \widehat{%
      \vrule\@width\z@\@height\ht\z@
      \vrule\@height\z@\@width\wd\z@}$}%
  \dp\tw@-\ht\z@
  \@tempdima\ht\z@ \advance\@tempdima2\ht\tw@ \divide\@tempdima\thr@@
  \setbox\tw@\hbox{%
    \raise\@tempdima\hbox{\scalebox{1}[-1]{\lower\@tempdima\box
        \tw@}}}%
  {\ooalign{\box\tw@ \cr \box\z@}}}
\makeatother

% Colorful Notes
\usepackage{color}
\definecolor{Red}{rgb}{1,0,0}
\definecolor{Blue}{rgb}{0,0,1}
\definecolor{Purple}{rgb}{.5,0,.5}
\def\red{\color{Red}}
\def\blue{\color{Blue}}
\def\gray{\color{gray}}
\def\purple{\color{Purple}}

\newcommand{\rnote}[1]{{\red [#1] }} % \rnote{foo} then 'foo' is red
\newcommand{\bnote}[1]{{\blue #1}} % \bnote{foo} then 'foo' is blue
\newcommand{\gnote}[1]{{\gray [#1]}} % gray note  -- for comments
\newcommand{\pnote}[1]{{\purple [#1]}} % green note  -- for comments

% Math Environments
\newtheorem{theorem}{Theorem}
\newtheorem{definition}{Definition}
\newtheorem{assumption}{Assumption}
\newtheorem{lemma}{Lemma}
\newtheorem{remark}{Remark}
\newtheorem{prop}{Proposition}
\newtheorem{corollary}{Corollary}
\newtheorem{example}{Example} 
\newtheorem{question}{Question}
\newtheorem{claim}{Claim} 
\newtheorem{conjecture}{Conjecture} 

% Custom Math Commands
\newcommand{\vt}{\vskip 5mm} % vertical space
\newcommand{\fl}{\noindent\textbf} % first line
\newcommand{\Fl}{\vt\noindent\textbf} % first line with space above
\def\R{\mathbb{R}} % Real numbers
\def\Z{\mathbb{Z}} % Integers
\def\E{\mathbb{E}} % Expectation
\def\P{\mathbb{P}} % Probability
\def\Q{\mathbb{Q}} % Q probability
\newcommand{\indep}{\perp \!\!\! \perp}  %independence symbol
\newcommand{\ifelse}[4]{\left\{ \begin{array}{l@{\quad:\quad}l}#1&#3\\#2&#4\end{array}\right.}
\newcommand{\curly}[2]{\left\{ \begin{array}{l@{\quad}l}#1\\#2\end{array}\right.}
\newcommand{\di}{\left. \mathrm{Di} \right|}
\newcommand\norm[1]{\left\lVert#1\right\rVert}

\usepackage[utf8]{inputenc}
\usepackage{hyperref}
\usepackage{amsmath}
\usepackage[shortlabels]{enumitem}
\newtheorem{problem}{Problem}

% Concise Vector Macro: write bracket vectors of arbitrary length
% Usage: \columnvector{x,y,1} or \columnvector{2,x^2}
\usepackage{stackengine}
\newcommand\columnvector[1]{\setstackEOL{,}\bracketVectorstack{#1}}


\title{Lecture Notes for Mihaela Ifrim's Course on Dispersive PDE}
\date{Fall 2022}


\begin{document}
\title{Statistics }
\author{Max Hill}
\maketitle
\tableofcontents
\newpage

\section{Lecture 1: 2022-09-08}
\subsection{Introduction to Dispersive PDEs}
\begin{definition}[Dispersive PDE]
  Informally, a PDE is characterized as \textbf{dispersive} if, when the the
  boundary conditions are dropped, its wave solutions are going to spread out in
  space as time evovles. Like a rock thrown into water.
\end{definition}

For now we will focus on \textit{linear} dispersive PDEs. Associated with a
dispersive PDE:
\begin{enumerate}
  \item Dispersive Estimates
  \item On the Fourier side, different frequencies at different speedss in
  different directions.
\end{enumerate}
We first consider the simplest example.

\begin{example}[Linear dispersive PDE with constant coefficients] \label{ex:linear-dispersive-pde-with-constant-coefficients}
  Suppose $u(t,x):\R\times\R^{d}\to V \in \R^{d}$ such that
  \begin{equation}
    \label{eq:linear-constant-coefficient-pde}
    \curly{i\partial_{t}u(t,x)=Lu(t,x)}{u(0,x)=u_{0}(x)}
  \end{equation}
  where $L$ is a skew-adjoint constant coeffieicent differential operator of
  order $k$. In symbols, there exists constants $\left\{c_{\alpha}: \alpha\in
    \Z_{\geq 0}^{d}\right\}$ such that
  \begin{equation*}
    Lu(t,\cdot) = \sum_{|\alpha| \leq k}c_{\alpha} \partial_{x}^{\alpha}u(t,\cdot)
  \end{equation*}
  where $|\alpha| := \alpha_{1}+\ldots+\alpha_{d}$ and for $x\in \R^{d}$, the
  symbol $\partial_{x}^{\alpha}$ denotes the operator defined by
  \begin{equation}
    \label{eq:2} 
    \partial_{x}^{\alpha}u := \prod_{i=1}^{d} \partial_{x_{i}}^{\alpha_{i}}u
  \end{equation}
  So, for example, if $d=2$, $c_{(0,1)}=2$, $c_{(2,5)}=-3$, and
  $c_{\alpha}=0$ for all other choices of $\alpha$, then $L$ would be an order 7
  differential operator taking the following form:
  \begin{align*} 
    Lu(t,\cdot)
    &= 2 \partial_{x}^{(0,1)}u(t,\cdot)-3 \partial_{x}^{(2,5)}u(t,\cdot)\\
    &= 2\partial_{x_{1}}^{0}u(t,\cdot)\partial_{x_{2}}^{1}u(t,\cdot) -3\partial_{x_{1}}^{2}u(t,\cdot)\partial_{x_{2}}^{5}u(t,\cdot)
  \end{align*}
  or equivalently,
  \begin{equation*}
    Lu(t,x_{1},x_{2}) = 2u(t,x_{1},x_{2})u_{x_{2}}(t,x_{1},x_{2}) -3u_{x_{1}x_{1}}(t,x_{1},x_{2})u_{x_{2}x_{2}x_{2}x_{2}x_{2}}(t,x_{1},x_{2})
  \end{equation*}
  Writing $x,y$ in place of $x_{1},x_{2}$, we get the nicer-looking formulation:
  \begin{equation*}
    Lu(t,x) = 2u(t,x,y)u_{y}(t,x,y)-3u_{xx}(t,x,y)u_{yyyyy}(t,x,y).
  \end{equation*}
  Since the operator defined in \cref{eq:2} does not compose different partial
  differentiation operators together, $Lu$ does not involve any mixed partial derivatives of $u$ (e.g.
  you'll never see terms like $u_{xy}$ or $u_{xxy}$).
\end{example}

\begin{remark}
  The operator $L$ from
  \cref{ex:linear-dispersive-pde-with-constant-coefficients} is defined
  classically (i.e. pointwise) only if $u\in C^{k}(\R\times\R^{d})$, that is
  only if $u$ is $k$-times continuously differentiable. But of course we may
  extend $L$ in the usual manner so that it is defined in a distributional
  sense.)
\end{remark}

\subsection{Dispersion Relation}
\begin{definition}[Dispersion Relation and Frequency Operator]
  \label{def:dispersion-relation-frequency-operator}
  We can write $L$ in the form
  \begin{equation*}
    L = ia(D)
  \end{equation*}
  where $a$  is some function, called the \textbf{dispersion relation}, and $D$
  is the \textbf{frequency operator} defined as
  \begin{equation*}
    D = \frac{1}{i}\nabla := \left( \frac{1}{i}\partial_{x_{1}},\ldots,\frac{1}{i}\partial_{x_{d}} \right) 
  \end{equation*}
\end{definition}
\begin{remark}[Polynomial Dispersion Relations]
  \label{rmk:polynomial-dispersion-relation}
  It turns out that the form of the dispersion relation is important. When the
  operator $L$  is of the form
  \begin{equation*}
    Lu= \partial_{t}u
  \end{equation*}
  (or something), then the dispersion relation
  takes the form
  \begin{equation}
    \label{eq:polynomial-dispersion-relation}
    a(\xi_{1},\ldots,\xi_{d})= \sum_{|\alpha| \leq k}i^{|\alpha|-1}c_{\alpha}\xi_{1}^{\alpha_{1}}\cdots \xi_{d}^{\alpha_{d}}
  \end{equation}
\end{remark}
A large number of PDEs are governed by dispersion relations of the form given
in \cref{eq:polynomial-dispersion-relation}, as we show in the next example.

\begin{example}[Examples for \cref{rmk:polynomial-dispersion-relation}]
  Here we list some examples of PDEs whose dispersion relations take the form
  shown in \cref{eq:polynomial-dispersion-relation}.
  \begin{itemize}
    \item \textbf{A degenerate (i.e. nondispersive) example.} Suppose
    \cref{eq:linear-constant-coefficient-pde} takes the form
    \begin{equation*}
      \curly{\partial_{t}u(t,x) = iwu(t,x)}{u(0,x)=u_{0}(x)}
    \end{equation*}
    where $u:\R\times\R\to\R$ and $w\in \R$. In this case the dispersion
    relation is
    \begin{equation*}
      a(\xi) = w.
    \end{equation*}
    The solution to this system is $u(t,x)= u_{0}e^{itw}.$
    \item \textbf{Another degenerate equation.} Fix $\nu\in \R^{d}$ and
    consider the
    \begin{equation*}
      \curly{\partial_{t}u(t,x) = -\nu\cdot \nabla_{x}u(t,x)}{u_{0}(0,x)=u_{0}(x)}
    \end{equation*}
    An example solution to this is
    \begin{equation*}
      u(t,x) = u_{0}(x-\nu t)
    \end{equation*}
    something something transport equation. 

    \item \textbf{The Airy Equation.} Suppose $d=1$, $u=u(t,x):\R\times\R \to \R$ such
    that
    \begin{equation*}
      \partial_{t}u +\partial_{xxx}u=0.
    \end{equation*}
    In this case, the dispersion relation is
    \begin{equation*}
      a(\xi) = \xi^{3},\quad (\xi\in \R).
    \end{equation*}

    \item \textbf{Schrodinger (?) equation.} Suppose
    \begin{equation*}
      i\partial_{t}u +\Delta u = 0.
    \end{equation*}
    In this case, $L=-\Delta$ and $a(\xi)=|\xi|^{2}$. 
    \item \textbf{Wave Equation.} Suppose
    \begin{equation}\label{eq:wave-equation}
      -u_{tt}+\Delta u = 0.
    \end{equation}
    in any dimnensions. This is dispersive.
    \item \textbf{Klein-Gordon Equation.} Suppose
    \begin{equation*}
      -u_{tt} + \Delta u - u = 0.
    \end{equation*}
    This is dispersive.
  \end{itemize}
\end{example}

It is not always the case that dispersion relations are polynomials of the form
given in \cref{eq:polynomial-dispersion-relation}. Indedd, for deep water
gravity waves, $a(\xi)= |\xi|^{1/2}$. Also for capillary waves, something
something. Also the BO equation
\begin{equation*}
  u_{t}+ H \partial_{x}^{2}u = 0
\end{equation*}
has dispersion relation $a(\xi)=\xi|\xi|$.
\subsection{Group Velocity}
\begin{definition}[Wave-Plane Solution]
  \label{def:wave-plane-solution}
  A \textbf{wave-plane solution} is the name for functions of the form
  \begin{equation}
    \label{eq:wave-plane-solution}
    u(t,v) = e^{i(kx-wt)}.
  \end{equation}
\end{definition}
\begin{definition}[Wave Number and Angular Frequency]
  \label{def:wave-number-angular-frequency}
  The parameter $k$ in \cref{eq:wave-plane-solution} is called the \textbf{wave
    number}. The parameter $w$ is called the \textbf{angular frequency}.
\end{definition}
\begin{example}[Wave-plane solutions for the wave equation]
  If we plug the function $u$ from \cref{eq:wave-plane-solution} into
  the wave equation shown in \cref{eq:wave-equation}, we obtain
  \begin{equation*}
    -(-iw)^{2}e^{(i(kx-wt))} +(ik)^{2}e^{(i(kx-wt))} =0
  \end{equation*}
  which implies that $w^{2}=k^{2}$, and hence that $w(k) = \pm |k|.$
\end{example}
\begin{definition}[Group Velocity]
  The derivative of the dispersion relation with respect to the dwave number is
  called the \textbf{group velocity}.
\end{definition}
\begin{definition}[Dispersive Equation]
  A PDE is said to be dispersive if $a''(\xi) \neq 0$. 
\end{definition}

\subsection{Symmetries for Linear Dispersive PDEs}
\begin{enumerate}
  \item All are invariant under time and space translations. That is, suppose
  $\tau$ is the translation operator defined by
  \begin{equation*}
    \tau u(t,x) = u(t-t_{0},x-x_{0})
  \end{equation*}
  for some fixed $t_{0}$ and $x_{0}$. Then $\tau u$ is a solution provided that
  $u$ is a solution. 
  \item Scaling symmetries. For any $\lambda>0$,  the function
  \begin{equation*}
    u(t/\lambda^{k},x/\lambda)
  \end{equation*}
  is a solution whenever $u(t,x)$ is a solution.
  \item I think there were more, but my hand got tired and I stopped taking notes.
\end{enumerate}

\section{Lecture 2: 2022-09-13}
\subsection{Full Dispersion}
Recall the dispersion relation $a(\xi)$. If $a''(\xi) \neq 0$ then the PDE is
said to be \textbf{fully dispersive}. If $\xi\in \R^{d}$ then $\nabla^{2}a(\xi)$
is the hessian. The appropriate thing to do in that case is to diagonalize...
we'll get to that.

\begin{example}[Wave Equation]
  Suppose we have $\square u = 0$ with $d=1$. Then $a(\xi) = |\xi|$ so that
  $a''(\xi)=0$ which is not dispersive. But if $d>1$ then since
  $\alpha(\xi)=|\xi|$. So that
  $a'(\xi)= -\nabla a(\xi)= \frac{\xi}{\left| \xi \right|}$. Then the hessian is
  \begin{equation*}
    \nabla^{2}_{\xi} a(\xi) 
    = \frac{1}{\left| \xi \right|}\left[ I_{n}-\frac{\xi}{|\xi|}\otimes \frac{\xi}{|\xi|} \right] 
  \end{equation*}
  where $I_{n}$ is the $n\times n$ identity matrix. The thing in brackets is the
  orthogonal projection onto the plane perpendicular to $\xi/|\xi|$ (i.e.
  tangent to the sphere).

  For example, consider $d=2$. Then we can diagonalize the Hessian matrix
  $\nabla^{2}_{\xi}a$.  You get $\lambda=0$ and $\lambda= \frac{1}{|\xi|}$ as
  eigenvalues and the diagonalized matrix is
  \begin{equation*}
    \begin{pmatrix}
      0&0\\
      0&\frac{1}{|\xi|}
    \end{pmatrix}
  \end{equation*}
  The interpretation is that since one of the eigenvalues is zero, so you don't
  have full dispersion. It's like degenerate dispersion. In particular, we don't
  have disperion in the radian direction. Try playing with this in the 2D. 
\end{example}
\begin{remark}
  The wave equation in $d>1$ has dispersion (degenerate). Also, last lecture had
  something false in it. The false claim was that finite speed of propagation is
  impied by bounded group velocity. (Recall that the \textbf{group velocity} is
  the negative of the derivative of the dispersion relation: $-\nabla a(\xi)$.) 
\end{remark}
\begin{example}
  Suppose $u:\R\times\R^{3}\to \R$ satisfying
  \begin{equation}\label{eq:1}
    \curly{i\partial_{t}u +A(D)u =0}{u_{0}=u(0,\lambda)}
  \end{equation}
  We claim that this equation with $A(D)=|D|$ does not have finite speed of
  propagation.

  Finite speed propagation is if $u_{0}$ is supported on $B(a,R)$ then $t>0$,
  $u(x,t)$ is going to be supported by $B(a,R+ct)$. The infimum of all such $c$'s
  satifying the above condition is called \textbf{the finite speed of
    propagation.}

  Recall that we can take the Fourier transform of \cref{eq:1} and solve for the
  fundamental solution:
  \begin{equation*}
    e^{it|D|} = \cos(t|D|)+ i \sin(t|D|)
  \end{equation*}
  here $D$ is the derivative with radial.
  \begin{equation*}
    |D| \frac{\sin(t|D|)}{|D|}
  \end{equation*}
  Recall the fundamental solution to the wave equation is
  $K_{\square}= c \frac{1}{t}\delta$. Give this a little bit of thought. Do this
  example to bush up. This example shows why the finite group velocity does not
  imply finite speed of propagation. This example looks like the example from
  last time that had a star with it. The operatior $|D|$ is the operator with
  symbol $|\xi|$. In one dimension, this is the Hilbert transform.
\end{example}


\subsection{Long-Time Dynamics for Linear Dispersive Waves}
\Fl{Question:} What are the long time dynamics for linear dispersive waves?


Scalar case. Suppose we have
\begin{equation*}
  \curly{i\partial_{t}u +A(D)u=0}{u_{0}(x)=u(0,x)\in \mathcal{S}}
\end{equation*}
where $\mathcal{S}$ denotes the Schwarz space. What happens with this wave as
$t \to \infty$?

Then the solution is
\begin{equation*}
  \hat{u}(t,\xi) = \hat{u}_{0}e^{it a(\xi)}
\end{equation*}
where $a(\xi)$  is the symbol of the operator $A$. Remember, dispersivity is
that every frequency moves in its own direction and with its own velocity. We
want to quantify this. My solution:
\begin{equation}\label{eq:3}
  \hat{u}(t,\xi) = \hat{u}_{0}(\xi)e^{ita(\xi)}
\end{equation}
now let's go back and write it on the spatial side:
\begin{equation*}
  u(t,x) = \int e^{ix\cdot\xi}e^{ita(\xi)}\hat{u}_{0}(\xi)d\xi
\end{equation*}
where $\hat{u}_{0}(\xi)$ is a Schwarz function (since $u_{0}\in \mathcal{S}$ and
the Fourier Transform maps Schwarz functions to Schwarz functions). Therefore
\begin{equation*}
  u(t,x) = \int e^{i(x\cdot\xi+ta(\xi))}\hat{u}_{0}(\xi)d\xi
\end{equation*}
Heuristically, we make the assumption that the waves (each localized a different
frequency) travel and go out in a linear fasion (at least when time is big).
When $t$ is very large, we write
\begin{equation*}
  u(t,x) = \int e^{it \left[ \frac{x}{t}\xi+a(\xi) \right] }\hat{u}_{0}(\xi)d\xi
\end{equation*}
and we think of $v:=x/t$ as velocity. Thus
\begin{equation*}
  u= u(t,v) = \int e^{it \left[ v\xi+a(\xi) \right] }\hat{u}_{0}(\xi)d\xi
\end{equation*}
and this is an oscillatory integral. This is oscillate rapidly when $t$ is
large. One nice thing about it is that it has a complex phase, which is what
makes it oscillate. We are interested in the long-time dynamics. That will
result in cancellations, which can be seen when integrating by parts. There is a
complication when doing integration by parts however because we don't know that
the quantity 
\begin{equation*}
  \frac{\partial }{\partial \xi} \left[ itv\xi +a(\xi) \right] 
\end{equation*}
is nonzero and integration by parts would require it to be on the denominator.
So we need to introduce something called stationary/nonstationary phase. In
$d=1$, this is from Stein. The main understanding is the following:

Let
\begin{equation*}
  I = \int e^{i\lambda\phi(\xi)}a(\xi)d\xi 
\end{equation*}
First, suppose that $\phi(\xi) \neq 0$. This is called the nonstationary phase
argument. Take
\begin{equation*}
  I 
  = \int \partial_{\xi} \left( e^{i\lambda \phi} \right)\frac{1}{i\lambda \phi'(\xi)} a(\xi)d\xi  
\end{equation*}
Integrating by parts $N$ times will give a $\lambda^{-N}$ in front (e.g. if $a$
is a polynomial then $N$ would be the degree of $a$). Therefore as
$t\to \infty$,
\begin{equation*}
  I = O(\lambda^{-N})
\end{equation*}
which decays very quickly. Therefore the solution of \cref{eq:3}, understood as
a function $u=u(t,v)$ disperses very fast---it decays rapidly.

On the other hand, suppose that $\phi'(\xi) = 0$. For example, if $\phi(\xi)$ is
a constant. But that doesn't makeif $\xi_{0}$ is a critical point sense for our
problem. So let's correct for that by requiring that
\begin{equation*}
  \phi''(\xi) \neq 0.
\end{equation*}
Thus any zeros are when we have critical points of $\phi$. Around those points,
we cannot integrate by parts. But what can we do? Another heuristic: if
$\xi_{0}$ is a critical point, assume that when $\xi\approx \xi_{0}$, we can
Taylor expand:
\begin{align*}
  \phi(\xi) 
  &= \phi(\xi_{0}) + (\xi-\xi_{0})\phi(\xi_{0}) + \frac{1}{2}(\xi-\xi_{0})^{2}\phi''(\xi) \\
  &= \phi(\xi_{0}) + \frac{1}{2}(\xi-\xi_{0})^{2}\phi''(\xi).
\end{align*}
so that
\begin{align*}
  I 
  &\approx \int e^{i\lambda \left[ \phi(\xi_{0})+\frac{1}{2}(\xi-\xi_{0})^{2}\phi''(\xi_{0}) \right] } a(\xi_{0})d\xi \\
  &= e^{i\lambda \phi(\xi_{0})}a(\xi_{0}) \int e^{\frac{1}{2}i\lambda (\xi-\xi_{0})^{2}\phi''(\xi_{0})}d\xi
\end{align*}
and the integral is a complex Gaussian which can be computed with the rule
\begin{equation*}
  \int_{0}^{\infty}e^{i\alpha x^{2}}dx 
  = e^{i\pi \text{sign}(|\alpha|)/4} \sqrt{\frac{\pi}{4\alpha}}
  , \quad \alpha  \neq 0.
\end{equation*}
with $\alpha = \frac{1}{2}\lambda \phi''(\xi_{0})$. So we get
\begin{equation*}
  I 
  \approx e^{i \lambda \phi(\xi_{0})} a(\xi_{0}) \frac{1}{\sqrt{\lambda}} \cdot \frac{1}{\sqrt{\phi''(\xi_{0})}}
\end{equation*}
multiplied by some other thing that aren't important. Taking $\phi = v\xi +
a(\xi)$  and $\lambda= t$ gives
\begin{equation*}
  u(t,vt) 
  \approx e^{it \left[ v_{\xi}\xi_{0}+ a(\xi_{0}) \right] }\frac{1}{\sqrt{t}}\hat{u}_{0}(\xi_{v})
\end{equation*}
with some other things.

\section{2022-09-15: Lecture Notes}

\subsection{Littlewood-Payley Decomposition}
Let $\phi$ be a smooth radial\footnote{A function $\phi$ is \textbf{radial} iff
  $\phi(x)=\phi(|x|)$ for all $x$.} function such that
\begin{equation*}
  \curly{\text{supp($\phi$ )}\subseteq \left\{\xi\in \R^{n}: 0 \leq |\xi| \leq
      2\right\}}{\phi\equiv 1 \text{ in }B(0,1/2)}
\end{equation*}
where $B(0,1)\subset \R^{n}$ denotes the unit ball centered at the origin. In
addition, for each $\xi\in \R^{n}$ define
\begin{equation*}
  \psi(\xi):= \phi(\xi)- \phi(2\xi).
\end{equation*}
It is easy to see that $\psi$ is a radial function with
\begin{equation}
  \label{eq:support-of-psi}
  \text{supp}(\psi) \subseteq \left\{\xi\in \R^{n}: \frac{1}{2} \leq |\xi| \leq 2\right\}.
\end{equation}
We then construct a sequence of functions in the following manner. For each $k\in
\Z$,  define $\phi_{k}$ by 
\begin{equation*}
  \psi_{k}(\xi):= \psi\left(\frac{\xi}{2^{k}}\right), \quad \xi\in \R^n.
\end{equation*}
Then the sequence of functions $(\psi_{k})_{k\in\Z}$ is a \textbf{partition of
  unity}. To see this, observe that
\begin{align*}
  J(\xi)
  &:= \left\{k\in \Z : \psi_{k}(\xi) \neq 0 \right\}\\
  &= \left\{k\in \Z : \frac{1}{2} \leq |\xi/2^{k}| \leq 2\right\} \\
  &= \left\{k\in \Z : -1 \leq \log_{2}|\xi| - k \leq 1\right\} \\
  &= \left\{k : \log_{2}|\xi|-1 \leq k \leq \log_{2}|\xi| +1 \right\}
\end{align*}
which is clearly a finite set, and, letting $j:=\lceil \log_{2}|\xi| \rceil-1$,  observe that
\begin{align*}
  \sum_{k\in \Z} \psi_{k}(\xi) 
  &= \sum_{k\in J}\psi_{k}(\xi)\\
  &= \psi(\xi/2^{j}) + \psi(\xi/2^{j+1})\\
  &= \phi(\xi/2^{j}) - \phi(\xi/2^{j-1})\\
  &= 1-0\\
  &= 1
\end{align*}
\begin{definition}[Littlewood-Paley Projection]
  \label{def:littlewood-paley-projection}
  For each $k\in \Z$, let $P_{k}$ be the Fourier multiplication operator defined
  on $L^{2}(\R^{n})$ by the formula
  \begin{equation*}
    \widehat{P_{k}f}(\xi)
    := \psi \left( \xi/2^{k}\right)\hat{f}(\xi)
  \end{equation*}
  and define $P_{ \leq k}$ by
  \begin{equation*}
    \widehat{P_{ \leq k}f}(\xi) := \sum_{\ell \leq k} \widehat{P_{\ell}f}(\xi)= \phi(\xi/2^{k})\hat{f}(\xi)
  \end{equation*}
  for all $f\in L^{2}(\R^{n})$.
\end{definition}
\begin{remark}
  The operators $P_{k}$ and $P_{\leq k}$ are \textbf{almost} projections,  since
  $\psi(\xi/2^{k})$ is a smooth approxmation of an indicator function (but the
  tail turns out not to be an issue).
\end{remark}
\begin{lemma}[Littlewood-Paley Projection Properties]
  \label{lem:littlewood-paley-projection-properties}
  The following properties hold for all $f\in L^{2}(\R^n)$:
  \begin{itemize}[(i)]
    \item $P_k= P_{\leq k}-P_{\leq k-1}$ 
    \item $\lim_{k \to -\infty}P_{\leq k}=0$ and $\lim_{k \to \infty} P_{\leq
      k}f = f$ in $L^{2}$ 
    \item $\sum_{k\in \Z}P_{k}f = f$ in $L^{2}$ 
  \end{itemize}
\end{lemma}
\begin{remark}
  Property (iii) of \cref{lem:littlewood-paley-projection-properties} holds if
  $f\in L^{p}$ but not in general if $f\in L^{1}_{loc}$. Consider the case in
  which $f=1$ and $\hat{f}=\delta_{0}$. Then
  $\text{supp}(\hat{f})=\left\{0\right\}$.
\end{remark}
\subsection{Physical Space}
How are we to interpret $P_{k}$ in physical space?
\begin{definition}[Dilation Operator]
  \label{def:dilation-operator}
  For each $\lambda>1$ and each $1 \leq p \leq \infty$, define an operator
  $\di_{\lambda}^{p}$ on $L^{2}$ by
  \begin{equation*}
    \di_{\lambda}^{p}(f)(x):= \lambda^{-n/p}f(x/\lambda)
  \end{equation*}
  for each $x\in \R^n$ and each $f\in L^{2}(\R^n)$. 
\end{definition}
\begin{lemma}[Dilation Properties]
  \label{lem:dilation-properties}
  The following properties hold:
  \begin{itemize}
    \item If $\mathcal{F}$ is the Fourier transform operator, then
    \begin{equation*}
      \mathcal{F}\di_{\lambda}^{p} = \di_{\lambda^{-1}}^{q}\mathcal{F}
    \end{equation*}
    where $\frac{1}{p}+\frac{1}{q}=1$.
    \item $\phi(\xi/2^{k})= \di_{2^{k}}^{\infty}\phi(\xi)$ 
    \item $P_{\leq k}f(x)=\di_{2^{-k}}^{1}\hat{\phi}*f(x) = \int_{\R^m}f(x-y)2^{k}\hat{\phi}(2^{k}y)dy $
  \end{itemize}
\end{lemma}
\begin{remark}
  I didn't really understand this remark. 
\end{remark}
\begin{lemma}
  \label{lem:littlewood-paley-Lp-lemma}
  Let $k\in \Z$ and let $f$ be a function such that
  $\text{supp}(\hat{f}) \subseteq \left\{\xi: 2^{k-1}\leq |\xi| \leq
    2^{k+1}\right\}$. Then
  \begin{equation*}
    \norm{\nabla f}_{L^{p}} \sim 2^{k} \norm{f}_{L^{p}}, \quad 1 \leq  p\leq \infty
  \end{equation*}
  and in particular,
  \begin{equation*}
    \norm{P_{k}f}_{L^{p}} \sim 2^{k} \norm{f}_{L^{p}}
  \end{equation*}
  in particular we have
  \begin{equation}\label{eq:8}
    \norm{\nabla P_{k}f}_{L^{p}} \sim 2^{k} \norm{P_{k}f}_{L^{p}}
  \end{equation}
\end{lemma}
\section{2022-10-04: Lecture Notes}
\begin{proof}[Proof of \cref{lem:littlewood-paley-Lp-lemma}]
  From the notes of last time, we have
  \begin{equation*}
    \norm{\nabla f(x)}_{p} \leq  2^{k}\norm{f}_{p}
  \end{equation*}
  It remains to prove an inequality in the other direction. Intuitively, we need to ``invert'' $\nabla$. Recall
  we have
  \begin{equation*}
    \widehat{\partial_{x_{j}}f}(\xi) = 2\pi i\xi_{j} \widehat{f}(\xi)
  \end{equation*}
  Therefore if $|\xi| \leq  2^{k+2}$, we have
  \begin{equation*}
    \phi\left(\frac{\xi}{2^{k+2}}\right)\widehat{\partial_{x_{j}}f}(\xi) = 2\pi i \xi_{j}\widehat{f}(\xi)
  \end{equation*}
  then we multiply both sides by $\xi_{i}$ and sum over $j$:
  \begin{equation*}
    \sum_{j=1}^{m} \xi_{j}\phi\left(\frac{\xi}{2^{k+2}}\right)\widehat{\partial_{x_{j}}f}(\xi) 
    = \sum_{j=1}^{m} 2\pi i \widehat{f}(\xi)\xi^{2}_{j}|\xi|^{2}
    = 2\pi i \hat{f}(\xi)|\xi|^{2}
  \end{equation*}
  here $m$ is the dimension ($\xi\in \R^m$). Then (why)
  \begin{equation*}
    \widehat{f}(\xi) 
    = \sum_{j=1}^{m} \frac{\xi_{j}\phi \left( \frac{\xi}{2^{k+2}} \right) \widehat{\partial_{x_{j}}f}(\xi) }{2\pi i |\xi|^{2}} 
  \end{equation*}
  Then taking the inverse Fourier transform and using that $\widehat{f}\cdot\widehat{g}=
  \widehat{f*g}$
  \begin{equation}\label{eq:4} 
    f = 2^{-k}\sum_{j=1}^{m}K_{k,j}* \delta_{x_{j}}f 
  \end{equation}
  where
  \begin{align*}
    K_{k,j} 
    &= 2^{k}\int \phi \left( \frac{\xi}{2^{k+2}}
      \right)\frac{\xi_{j}}{2\pi i |\xi|^{2}} e^{2\pi i x \cdot \xi} d\xi\\
    &=2^{nk} \int \phi \left( \frac{\xi}{2^{2}} \right) \frac{\xi_{j}}{2\pi i |\xi|^{2}}e^{\pi i 2^{k}\cdot x\cdot \xi}d\xi 
  \end{align*}
  where the second equality follows by a change of variable. In particular, we
  have
  \begin{equation*}
    \left| K_{k,j}(x) \right| \lesssim 2^{nk}
  \end{equation*}
  where the squiggly inequality hides some constants. Note that we can write our
  formula as
  \begin{align*}
    K_{k,j}(x) 
    &=2^{nk} \int \phi \left( \frac{\xi}{4} \right)\frac{\xi_{j}}{2\pi i |\xi|^{2}}\partial_{\xi} \left( \frac{e^{2 \pi i 2^{k}}x\cdot\xi}{2 \pi i 2^{k} x} \right)   ds\\
  \end{align*}
  which suggests an integration by parts. Integrating by parts $s$ times, we obtain
  \begin{equation*}
    \left| K_{k,j}(x) \right| \lesssim 2^{nk} \left| 2^{k}x \right|^{-s}
  \end{equation*}
  for $s>0$. Here $K_{k,j}$ is like an approximation of identity. Finally,
  applying the Minkowski inequality to \cref{eq:4}, we get
  \begin{equation*}
    \norm{\nabla f}_{L^{p}}\geq 2^{k} \norm{f}_{L^{p}},
  \end{equation*}
  as required.
\end{proof}

\begin{remark}[Singularity at zero]
  There is some question about a singularity at zero. But this is not an issue
  since under the hypotheses of \cref{lem:littlewood-paley-Lp-lemma}, the
  function $\hat{f}$ is zero in a neighborhood of zero. 
\end{remark}

\begin{remark}
  We will use the notation $f_{k}:=P_{k}f$. Morally at the level of $L^{p}$,
  $ \norm{\nabla f_{k}}_{p}\sim \norm{f_{k}}_{L^{p}}$ so heuristically,
  $\nabla\sim \sum_{k}2^{k}P_{k}$. This might become clear later.
\end{remark}

Now we want to connect $P_{k}(f)$ on $P_{ \leq k}f$ to $f$ itself. We have
\begin{enumerate}
  \item
  $\norm{P_{ \leq k}f}_{L^{p}}\leq \int_{\R^m} \norm{f(x-2^{-k}y)}_{L^{p}}
  \left| \widehat{\phi}(y) \right|dy \lesssim \norm{f}_{L^{p}}$ also
  \rnote{check this}
  $\widehat{P_{ \leq k}}\widehat{f} = \phi(\xi/2^{k})\hat{f}(\xi)$ implies
  \begin{equation}\label{eq:6}
    \norm{P_{ \leq k}f}_{L^{p}} \leq \norm{f}_{L^{p}}
  \end{equation}
  \item (Cheap LP inequality): By Minkowski inequality, since
  \begin{equation*}
    f= \sum_{k} P_{k}f 
  \end{equation*}
  implies
  \begin{equation*}
    \sup_{k} \norm{P_{k}f}_{p} \lesssim \norm{f}_{p} \leq \sum_{k}\norm{P_{k}f}_{p} 
  \end{equation*}
\end{enumerate}

There is a way to elegantly prove the Sobolev embeddings using Littlewood-Paley.
The idea will be to prove it for a dyadic piece.
\begin{lemma}[Non-endpoint Sobolev Embedding]
  \label{lem:non-endpoint-sobolev-embedding}
  Let $1 \leq p<q \leq \infty$ such that $\frac{1}{p}-\frac{1}{n}>\frac{1}{2}$.
  Then
  \begin{equation}\label{eq:5}
    \norm{f}_{L^{q}(\R^n)} \leq C_{p,q,n} \norm{f}_{L^{p}(\R^n)} + \norm{\nabla f}_{L^{p}(\R^n)}
  \end{equation}
  for all $f\in L^{p}(\R^n)$ such that the right hand side is finite. Here,
  $C_{p,q,n}$ is a constant which depends only on $p,q,$ and $n$. 
\end{lemma}
\begin{remark}
  The endpoint version is when $\frac{1}{p}-\frac{1}{n}=\frac{1}{2}$.  The proof
  will be similar but with some added complications. See Evans.
\end{remark}
\begin{proof}[Proof of \cref{lem:non-endpoint-sobolev-embedding}]
  Let $f$ be a Schwarz function, and denote the right hand side of \cref{eq:5}
  by $X$. Then by \cref{eq:6},
  \begin{equation}\label{eq:10}
    \norm{P_{k}f}_{p} \leq  X 
  \end{equation}
  for all $k$, and also
  \begin{equation*}
    \norm{\nabla P_{k}f}_{p} \leq \norm{\nabla f}_{p} \leq  X.
  \end{equation*}
  Also, by \cref{eq:8} from \cref{lem:littlewood-paley-Lp-lemma}, we have
  \begin{equation}\label{eq:9}
    \norm{P_{k}f}_{p} \lesssim 2^{-k} X
  \end{equation}
  By \cref{eq:9,eq:10}, we havem
  \begin{equation*}
    \norm{P_{k}f}_{ L^{p}} \lesssim \min \left\{1,2^{-k}\right\}X
  \end{equation*}
  Note taht if $|\xi|\sim 2^{k}$ then $2^{k-1} \leq |\xi| \leq 2^{k+1}$. What is
  the $L^{q}$ norm of $f$? Since $q>p$,  we use Bernstein's inequality, which
  states that
  \begin{equation*}
    \norm{P_{k}f}_{q} \leq 2^{\left( \frac{1}{p}-\frac{1}{q} \right)kn }\norm{P_{k}f}_{L^{p}}.
  \end{equation*}
  We will prove this proof later. It follows that
  \begin{equation*}
    \norm{P_{k}f}_{q} \lesssim 2^{\left( \frac{1}{p}-\frac{1}{q} \right)kn }\min \left\{1,2^{-k}\right\}X
  \end{equation*}
  then consideration of the cases where $k\to \pm \infty$, we find that the
  coefficient decays and the maximum value is when $k=0$. Then summing over
  $k$'s gives the result.
\end{proof}

\section{Lecture: 2022-10-06}
We start with the following inequality.
\begin{theorem}[Bernstein's Inequality]
  \label{thm:bernsteins-inequality-1}
  Let $f_{k}=P_{k}f$, and assume that $1 \leq p \leq q \leq \infty$. Then
  \begin{itemize}[(a)]
    \item The following inequality holds:
    \begin{equation*}
      \norm{f}_{L^{q}(\R^d)} 
      \lesssim 2^{k(\frac{d}{p}-\frac{d}{2})}\norm{f_{k}}_{L^{p}(\R^d)}.
    \end{equation*}
    \item A similar inequality holds for $f_{ \leq k}$
    \item For all $s\in \R$ and all $1 \leq p \leq \infty$,
    \begin{equation*}
      \norm{\left| \nabla \right|^{s}f_{k}}_{L^{p}}\sim 2^{ks}\norm{f_{k}}_{L^{p}}.
    \end{equation*}
  \end{itemize}
\end{theorem}
\begin{proof}[Proof of part (a) of \cref{thm:bernsteins-inequality-1}]
  Integrating by parts,
  \begin{equation*} 
    \norm{f_{k}}_{q} 
    = \norm{P_{k}f}_{q} 
    = \norm{f* 2^{kd}\widecheck{\psi}(2^{k}\cdot)}_{q} %%want to use widewidecheck\\
  \end{equation*}
  We'll use Young's inequality, which says that if
  $\frac{1}{p}+\frac{1}{r}=\frac{1}{q}+1$, then
  \begin{align*}
    \norm{g*h}_{q} 
    &\lesssim \norm{g}_{p}\norm{h}_{r}.
  \end{align*}
  Thus we obtain
  \begin{align*}
    \norm{f_{k}}_{q} 
    &\lesssim \norm{f}_{p}\norm{2^{kd}\widecheck{\psi}(2^{k}\cdot)}_{r}\\
    &\lesssim \norm{f}_{p}2^{k(d-\frac{d}{r})}\norm{\widecheck{\psi}}_{r}
  \end{align*}
  where the second step is by a change of variables $y \mapsto 2^{k}y $. Then
  \begin{equation*}
    \norm{f_{k}}_{q} \lesssim \norm{f}_{p}2^{kd(\frac{1}{p}-\frac{1}{q})}
  \end{equation*}
  To fix this, we will take a fatter Littlewood-Paley projection:
  \begin{equation*}
    \widetilde{P}_{k}:=P_{2^{k-2} \leq 2^{j} \leq 2^{k+2}}
  \end{equation*}
  (recall that $P_{k}$ was a projection to frequency $[2^{k-1},2^{k+1}]$).
  In this case, $\widetilde{P}_{k}P_{k}=P_{k}$ and we do the same computations as
  before to obtain
  \begin{equation*}
    \norm{f_{k}}_{q} = \norm{\widetilde{P}_{k}f_{k}}_{q}
  \end{equation*}
  Something:
  \begin{align*}
    \widehat{\widetilde{P}_{k}f(\xi)} = \left( \sum_{2^{k-2} \leq 2^{j}\leq 2^{k+2}} \psi_{2^{j}}  \right)(\xi)\widehat{f}(\xi) 
  \end{align*}
  which implies that
  \begin{align*}
    \widetilde{P_{k}}f
    &= f* \left( \sum\psi_{2^{j}}  \right)^{\widecheck{ }}\\ % fix this inverse widecheck
    &= f* \sum 2^{jd} \widecheck{\psi}(2^{j}\cdot )\\ 
    &\sim f* 2^{j }\sum \widecheck{\psi}(2^{j}\cdot) 
  \end{align*}
  this is simple but 
\end{proof}
I got distracted by latex not playing nice and missed something.

Recall we wanted to...

First the cheap LP inequality:
\begin{equation*}
  \sup_{k} \norm{P_{k}f}_{p} \lesssim \norm{f}_{p} \leq \sum_{k} \norm{P_{k}f}_{p} 
\end{equation*}
and for $p=2$, using Plancherel we can get
\begin{equation*}
  \norm{f}_{2} \sim \left( \sum_{k}\norm{P_{k}f}_{2}^{2}  \right)^{\frac{1}{2}} 
\end{equation*}
which we rewrite as
\begin{equation*}
  \norm{f}_{2} \sim \norm{\left( \sum_{k}\left| P_{k}f \right|^{2} \right)^{\frac{1}{2}} }_{2}.
\end{equation*}
We call this the ``square function''. Define
\begin{equation*}
  Sf = \left( \sum_{k} \left| P_{k}f \right|^{2} \right)^{\frac{1}{2}}
\end{equation*}
Then we have the following theorem
\begin{theorem}[LP inequality]
  Let $1<p<\infty$. Then
  \begin{equation*}
    \norm{Sf}_{p}\sim \norm{f}_{p}
  \end{equation*}
  where the constant depends on $p$. 
\end{theorem}
Sobolev Spaces. Suppose $f\in W^{s,p}$, and define
\begin{equation*}
  \norm{f}_{W^{s,p}}:= \sum_{j=1}^{s} \norm{\nabla^{j}f}_{p}<\infty 
\end{equation*}
where $s$ is an integer. Our goal is to replace $\nabla$ in terms of $P_{k}$.
We have the following lemma.

\begin{lemma}
  Let $j \geq 0$ and $1<p<\infty$. Then
  \begin{equation*}
    \norm{\nabla^{j}f}_{p} \sim \norm{\left( \sum_{k} \left| 2^{jk}P_{k}f \right|^{q} \right)^{\frac{1}{2}} }_{p}
  \end{equation*}
\end{lemma}
\begin{proof}
  Later.
\end{proof}

Then we can rewrite the norm.
\begin{equation*}
  \norm{f}_{W^{s,p}}\sim \norm{ \left(  \sum_{k}\left| (1+2^{k})^{s}P_{k}f \right|^{2} \right)^{2} }_{p}
\end{equation*}
actually in htis case, the thing is defined for any real number $s$, and not
just when $s$  is an integer. 

\begin{definition}[Japanese Bracket]
  \label{def:japanese-bracket}
  Define
  \begin{equation*}
    \left\langle x \right\rangle := \sqrt{1+|x|^{2}}.
  \end{equation*}
  This is called the \textbf{Japanese bracket}. We define an operator
  \begin{equation*}
    \left\langle \nabla \right\rangle^{s}:= \sqrt{1+\Delta^{s}}
  \end{equation*}
  which is defined in the Fourier side.
\end{definition}
\begin{remark}
  The ``plus 1'' in \cref{def:japanese-bracket} will take on significance when
  we distinguish between homogenoeous and inhomogenous Sobolev spaces.
\end{remark}
\begin{remark}
  The operator $\left\langle \nabla \right\rangle \approx |\Delta|$ at
  frequencies $|\xi|>1$ , but treats differently low frequencies.
\end{remark}
We have $\widehat{\nabla f}(\xi)=2\pi i \xi \hat{f}(\xi) $ and we define
similarly, $\widehat{|\nabla| f}(\xi)=2\pi i |\xi| \hat{f}(\xi) $  and the
Japanese bracket notation is defined similarly.

\section{2022-10-09: Make-Up Lecture}
In this lecture, we will discuss Stricharz estimates. These will be particular
to each dispersive equation. For example, nonlinear schrodinger equation has its
own Strichartz estimates which ar edifferent from, say,  those for the
Benjamin-Ono equtions or the wave equations. For your own equations, you'll need
to derive it.

\begin{remark}[Notation]
  For $t\in \R$ and $x\in \R^n$, denote
  \begin{equation*}
    \norm{u}_{L_{t}^{q}L_{X}^{r}}:= \left(  \int_{\R} \left[ \int_{\R^d}\left| u(t,x) \right|^{r}dx \right]^{\frac{1}{r}\cdot q}dt  \right) \frac{1}{q}
  \end{equation*}
\end{remark}
Let's consider the Schordinger equation
\begin{equation}\label{eq:nonlinear-schrodinger}
  \left\{ \begin{array}{l@{\quad\quad}l}
      i\partial_{t}u(t,x) +\Delta u(t,x) = 0 \\
      u(0,x)=u_{0}(x) 
    \end{array}\right.
\end{equation}
and for simplicity we will assume that $u_{0}(x)$ is a Shwartz function so that
$u(t,x)$ is Schwartz for all $x,t$.

Apply the Fourier transform in space to \cref{eq:nonlinear-schrodinger}.
\begin{equation*}\label{eq:transformed-nonlinear-schrodinger}
  \left\{ \begin{array}{l@{\quad\quad}l}
      i\widehat{u_{t}}(t,\xi) = |\xi|^{2}\widehat{u}(t,\xi)  \\
      \widehat{u}(0,\xi)=\widehat{u_{0}}(\xi) 
    \end{array}\right.
\end{equation*}
This is a seperable equation with respect to $t$. Separating gives:
\begin{equation*} 
  \frac{d\widehat{u}}{\widehat{u}} = -i |\xi|^{2}dt.
\end{equation*}
and integrating gives
\begin{equation*}
  \log \widehat{u} = -i |\xi|^{2}t+C
\end{equation*}
which ipmles that
\begin{equation*}
  \widehat{u}(t,\xi) = e^{-i |\xi|^{2}t}\widehat{u_{0}}(\xi)
\end{equation*}
and inverting the Fourier transform gives
\begin{align*}
  u(t,x) 
  &= \mathcal{F}^{-1}\left[ e^{-i|\xi|^{2}t} \widehat{u_{0}}(\xi) \right]\\
  &= \left( 2\pi \right)^{d/2}  \mathcal{F}^{-1}\left[ e^{-i|\xi|^{2}t}  \right]*u_{0}(x)
\end{align*}
Therefore it remains to compute $\mathcal{F}^{-1}\left[ e^{-i|\xi|^{2}t}\right]
$. We did this in a previous class.

We have a solution operator $S(t)=e^{it\Delta}$ \pnote{is this correct?} which is
\begin{align}\label{eq:7}
  S(t)u_{0}(x) 
  &= \left( 2\pi \right)^{d/2} \mathcal{F} \left[ e^{-t|\xi|^{2}i} \right]*u_{0}(x)\nonumber\\
  &= \left( 4\pi it \right)^{-d/2} \int_{\R^d}e^{\frac{i|x-y|^{2}}{4t}}u_{0}(y)dy 
\end{align}
The operator $e^{it\Delta}$ is the linear Schrodinger operator. Now we can
derive the dispersive estimate. First, observe that
\begin{align}\label{eq:A}
  \norm{u}_{\infty}
  &=\norm{e^{it\Delta}u_{0}}_{L^{\infty}(\R^d)}\nonumber\\
  &\lesssim |t|^{-\frac{d}{2}}\int_{\R^d}\left| u_{0}(y) \right| dy \nonumber\\
  &=\left| t \right|^{-\frac{d}{2}}\norm{u_{0}}_{L^{1}(\R^d)}
\end{align}
Second, observe that,
\begin{equation}\label{eq:B}
  \norm{e^{it\Delta}u_{0}}_{L^{2}}= \norm{u}_{L^{2}_{x}}^{2}= \norm{u_{0}}_{L^{2}_{x}}
\end{equation}
\pnote{somehow the second equality is obvious from \cref{eq:A}}
so that
\begin{equation*}
  \norm{u}_{H^{s}}=\norm{u_{0}}_{H^{s}} 
\end{equation*}
Interpolating \cref{eq:A,eq:B} gives, for all  $2 \leq p \leq \infty$, 
\begin{equation*}
  \norm{e^{it\Delta}}_{L^{p}\R^d}  \leq |t|^{-\left( \frac{d}{2}-\frac{d}{p} \right) }\norm{u_{0}}_{L^{p'}}
\end{equation*}
where $p'$ is the Holder conjugate of $p$.

Difference between $\dot{H}^{s}$  and $H^{s}$.
\section{2022-10-11: Lecture}
\begin{theorem}[Strichartz] \label{thm:strichartz}
  Assume we have $q,r$ a Stricharz admissible pair with $2 \leq q,r \leq
  \infty$.Then for $(q,r)$ and $(\tilde{q},\tilde{r})$ we have $(q,\infty,2)
  \neq (q,r,d)$, $\frac{2}{q}+\frac{d}{r}=\frac{d}{2}$. Then
  \begin{itemize}
    \item From Str. Est
    \begin{equation*}
      \norm{e^{i\Delta t}u_{0}}_{L^{q}L^{r}(\R\times \R^{d})}\lesssim C_{d,q,r} \norm{u_{0}}_{L^{2}_{x}}
    \end{equation*}
    \item From dual form of Str.: $\norm{\int_{\R}e^{is\Delta}F(s)ds
    }_{L^{2}_{x}}\lesssim_{d,\tilde{q},\tilde{r}}\norm{F}_{L^{\tilde{q}}L^{\tilde{r}}}$
    \item Inform str est
    \begin{equation*}
      \norm{\int_{t'<t}i(t-t')F(t')ds }_{L^{q}L^{r}}\lesssim_{d,q,r,\tilde{q},\tilde{r}}\norm{F}_{L^{\tilde{r}'L^{\tilde{q}'}}}
    \end{equation*}
  \end{itemize}
\end{theorem}
\begin{proof}
  $TT^{*}$ argument. Here $T:H \to B$,  $TT^{\infty}:B' \to B$,  $T^{*}:B' \to
  H'$ we have
  \begin{equation*}
    \norm{T}<\infty \iff \norm{TT^{*}}<\infty \iff \norm{T^{*}}<\infty
  \end{equation*}
  and here we have $T=e^{it\Delta}$.  We want to show that (1) $T:L^{2} \to
  L^{q}L^{r'}$. I will do the $TT^{*}$ bound. Here
  \begin{equation*}
    TT^{*}:L^{q'}L^{r'} \to L^{q}L^{r}
  \end{equation*}
  and we want to show this is truly the mapping domains and range. What is the
  adjoint of $T^{*}$? What do we usually do to find the adjoint? We look at the
  inner product and we move things. Our inner product is
  \begin{align*}
    \left\langle f,T^{*}G \right\rangle_{L_{t}^{2}L_{x}^{2}}
    &=\left\langle Tf,G \right\rangle_{L_{x}^{2}}\\
    &= \int \int e^{it\Delta}f\overline{G(t,x)}dxdt\\ 
    &= \int f \int \overline{e^{-it\Delta}G(t,x)}dxdt
  \end{align*}
  so that $T^{*} = \int e^{-it \Delta }G(t,x)dt $
\end{proof}
Using the dispersive estimate for the Schrodinger equation from earlier
\begin{align*}
  \norm{TT^{*}F}_{L^{q}_{t}L_{x}^{r}} 
  &= \norm{ e^{it \Delta}e^{-is \Delta}F} \\
  &= \norm{e^{i(t-s)\Delta}F}_{L^{q}L^{r}_{x}} \\
  &\lesssim \norm{\left| t-s \right|^{-\left( \frac{d}{2}-\frac{d}{r} \right) }\norm{F(s)}_{L^{r'}}ds}_{L^{q}_{t}} \\
  &= \norm{ \left| t \right|^{-2/q}* \norm{F(t)}_{L^{r'}}}_{L^{q}_{t}}
\end{align*}
at the end of the day, using some Littlewood Hardy Sobolev inequality, we get an
upper bound
\begin{equation*}
  \norm{F}_{L^{q'}L^{r'}}
\end{equation*}
and we get
\begin{equation*}
  \norm{TT^{*}}_{L^{q'}L^{r'}}\to L^{q}L^{r}<\infty
\end{equation*}
this proves the second part of \cref{thm:strichartz}. And first part follow from
something.

We will apply the following lemma with $K$ as the schrodinger operator.

\begin{lemma}[Christ-Kiselev]
  \label{lem:christ-kiselev}
  Suppose $X,Y$ are Banach spaces and $T:L^{p}(\R,X)\to L^{q}(\R,Y)$
  where $1 \leq p < q <\infty$ which is given by the integral transform
  \begin{equation*}
    Tf(t) = \int_{\R}K(t,s)f(s)ds 
  \end{equation*}
  where $K:\R\times\R \to \mathcal{S}(X,Y)$ 
\end{lemma}
\pnote{Need to fix this part.}

\subsubsection{Bilinear Strichartz estimates}
We have linear Sch equation  ($i \partial_{t}u = - \Delta u $). Two intial data
$v_{0},u_{0}$ both of which are scharz functions. \pnote{missed the last 15
  minutes of this lecture}


\section{Lecture: 2022-10-18}
At this point we all should know Duhamel's principle, which is what we did last
time. In general, for any PDE of the form
\begin{equation*}
  \left\{ \begin{array}{l@{}l}
      \partial_{t} u - Lu = N(u) \\
      u_{0} = u(0)
    \end{array}\right.
\end{equation*}
where $N(u)$ is nonlinear (and with the right $L$, which can be used for energy
estimates), Duhamel's principle states that your solution will be of the form
\begin{equation}\label{eq:duhamel-general}
  u(t,x) = \underbrace{e^{tL}\mu_{0}}_{\text{linear PDE}}+
  \underbrace{\int_{0}^{t}e^{(t-s)L}N(u(s))ds }_{\text{The Duhamel term.
      Inhomogenous. Call it $DN(u)$.}}
\end{equation}

When studying local wellposedness (LWP) for NLS (Nonlinear Schrodinger - semilinear). There is a huge
difference when studying LWP for quasilinear versus semilinear PDEs, and even
more so when the LPW theory is done in a low-regularity setting. In particular,
in the low regularity setting we cannot use the same techniques as were used in
the semilinear case. Thus the methods for proving LWP differ in these settings:

For the semilinear case, we use a \textbf{fixed point argument.} This
may work even in a low regularity setting (depending on the PDE you are
using). This technique may work for quasilinear equations, but only in the
high-regularity setting. The form of \cref{eq:duhamel-general} suggests a
perturbative method. We can reformulate our problem as
\begin{equation}\label{eq:11}
  u(t,x) = u_{\text{linear}} + DN(u)
\end{equation}

\begin{theorem}[Contraction mapping argument]
  Start with two Banach spaces $\mathcal{S}$ (not the Shwarz space) and
  $\mathcal{N}$. Suppose $D:\mathcal{N}\to \mathcal{S}$  is a linear operator
  such that there exists a positive constant $C_{0}>0$ such that
  \begin{equation*}
    \norm{DF}_{\mathcal{S}} \leq C_{0} \norm{F}_{\mathcal{N}}
  \end{equation*}
  for all $F\in \mathcal{N}$, and further suppose we have a nonlinear operator
  \begin{equation*}
    N:\mathcal{S} \to \mathcal{N}
  \end{equation*}
  such that $N(0)=0$ and such that $N$ satisfies the Lipschitz bound
  \begin{equation*}
    \norm{N(u)- N(v)}_{\mathcal{N}} \leq  \frac{1}{2C_{0}}\norm{u-v}_{\mathcal{S}}
  \end{equation*}
  where $u,v\in B_{\epsilon}= \left\{ u\in \mathcal{S}:\norm{u}_{\mathcal{S}}\leq \epsilon\right\}$ for $\epsilon>0$. Then
for all 
$u_{\rm lin}\in B_{\epsilon/2}$ there exists a unique solution
\begin{equation*}
  u\in B_{\epsilon}
\end{equation*}
which is a solution to \cref{eq:11}, with the map
\begin{equation*}
  u_{0} \mapsto u
\end{equation*}
is Lipschitz with constant at most $2$. In particular, we have
\begin{equation*}
  \norm{u}_{\mathcal{S}}\leq 2 \norm{u_{\rm lin}}_{\mathcal{S}}
\end{equation*}
\end{theorem}
\begin{example}[Nonlinear Schrodinger]
  Suppose we have the equation
  \begin{equation*}
    \left(i\partial_{t}+\frac{\Delta}{2}\right)u = |u|^{p-1}u
  \end{equation*}
  where $u:\R^{1+d}\to \mathbb{C}$, i.e. $u:[-T,T]*\times \R^{d}\to \mathbb{C}$.
  Then LWP in $e_{t}^{0}H^{s}\to \mathcal{N}$ in $L^{2}$ \pnote{missed something
  here}.
\end{example}

Consider the equation
\begin{equation}\label{eq:12}
  \left\{ \begin{array}{l@{}l}
      u_{t}+ \frac{1}{2}\Delta u = \mu|u|^{p-1}p \\
      u(t_{0},x)=u_{0}(x)\in H_{x}^{s}(\R^d)
    \end{array}\right.
\end{equation}
where we choose $p$ to be prime (e.g. $p=3$ is called the \textbf{cubic NLS}, $p=5$ is
the \textbf{quintic NLS}) as this often corresponds to something physcially relevant. If
$\mu=0$ there is no nonlinearity. If $\mu=+1$ or $\mu=-1$, these are called the
\textbf{focusing} and \textbf{defocusing} cases respectively. The focusing case
will not play a role for the rest of this lecture, but will come into play when
proving global well-posedness. If you want to prove global well-posedness you
have to show that (1) your solution in the nonlinear case looks like the linear
case, or (2) your solution splits into two bits: a soliton and a solution to
your linear probem. However, for LWP (the topic of this lecture), the sign of
$\mu$ doesn't make much a difference (because we'll be taking absolute values).

The nonlinearity in \cref{eq:12} doesn't look too bad because it's a polynomial,
but it's still a nonlinearity. We will study this problem on the real line, i.e.
we are looking for a solution
\begin{equation*}
  u:[-T,T]\times R^{d} \to \mathbb{C}
\end{equation*}
and we want to show that such a solution exists. Then if we can make $T$ large
enough, we'll have a global solution.

We need to determine the regularity of the space we will be working on. Terry
does a great job in Chapter 3 of discussing all the ways that you can treat this
problem. You can treat it classically, or that you have solutions in a
distributional sense, etc. There are multiple ways to think about this, so we
need to be precise about what we mean by local well-posedness. This problem has
many symmetries. To find scaling symmetry, we rescale our solutions
\begin{align*}
  &u(t,x) \to \lambda^{\alpha}u(t\lambda^{\beta},x\lambda^{\gamma})\\
  &u_{0}(x) \to  \lambda^{\alpha}u_{0}(x\lambda^{\gamma})
\end{align*}
where $\lambda>0$. For our problem in particular, we will use
\begin{align*}
  &u(t,x) \to \lambda^{-\frac{2}{p-1}}u\left(\frac{t}{\lambda^{2}},\frac{x}{\lambda}\right)\\
  &u_{0}(x) \to \lambda^{-\frac{2}{p-1}}u_{0}\left( \frac{x}{\lambda} \right) 
\end{align*}
so that 
\begin{equation*}
  \norm{u_{0}(x)}_{\dot{H}^{s_{c}}} = \norm{\lambda^{-\frac{2}{p-1}}u_{0} \left( \frac{x}{\lambda} \right)}_{\dot{H}^{s_{c}}}
\end{equation*}
where $s_{c}= \frac{d}{2}-\frac{2}{p-1}$. The critical value $s_{c}$ is the
threshold for the lowest regularity for which we can achieve a LWP result.

In this context, the terminology is as follows. The case in which $s>s_{c}$ is
called \textbf{subcritical}. Subcritical LWP theory is not that bad. The case
when $s=s_{c}$ is called \textbf{critical}. The case when $s<s_{c}$ is called
\textbf{supercritical}. Supercritical LWP theory is harder.

In general, showing LWP means showing
\begin{enumerate}
  \item Existence of a solution (e.g. by a contraction mapping argument for
  semilinear problems; different story for quasilinear problem) in
  $C_{t}^{0}[-T,T]H_{x}^{s}$.
  \item Uniqueness of solutions in the same regularity class (easy for both
  semi- and quasi-linear).
  \item Continuous dependence on the initial data (we call this continuous
  dependence, but when dealing with semilinear PDEs what we really need is
  Lipschitz continuity). That is, we want the map
  \begin{equation*}
    u_{0}\to u
  \end{equation*}
  to be continuous.
\end{enumerate}

Recall the Benjamin-Ono equation $u_{t}+H\partial_{x}^{2}u=uu_{x}$, which is
semilinear, but behaves more like a quasilinear problem in low regularity.

\begin{definition}[Well-Posedness]
  \label{def:local-well-posedness}
  We say the NLS is \textbf{locally well-posed (LWP)} in $H_{x}^{s}(R^{d})$ if
  for any $u^{*}_{0}\in H_{x}^{s}(\R^{d})$, there exists a time $T>0$ and an open
  ball $B\subset H_{x}^{s}$ containing $u_{0}^{*}$, and
  $X\subset C_{t}^{0}H_{x}^{s}([-T,T]\times \R^{d})$ such that for each $u_{0}\in B$
  there exists a strong unique solution $u\in X$ to the integral equation
  associated to hte NLS via Duhamel formulation, and furthermore, the map
  $u_{0}\to u$ is continuous from $B$ to $X$.
  \begin{itemize}
    \item If, in addition $ X=C_{t}^{0}H_{x}^{s}\left( [-T,T]\times \R^{d} \right) $ from this
    $H_{x}^{s}$-well posedness is \textbf{unconditional}. \pnote{This isn't correct?}
    \item If $T$ can be arbitrarily large, then we have \textbf{global
      well-posedness}. If $T$ only one $H_{x}^{s}$ of the $u_{0}$ then we say we
    hañve WP in the \textbf{subcritical sense}. \pnote{fix this}
    \item If $u_{0}\to u$ is uniformly continuous from $B$ to $X$ then we call
    it \textbf{uniform well-posedness}.
  \end{itemize}
\end{definition}


\section{Lecture Notes: 2022-10-20}
Recall
\begin{equation}\label{eq:nonlinear-schrodinger-equation}
  \left\{
    \begin{array}{l@{}l}
      iu_{t} +\frac{\Delta}{2}u=\left| u \right|^{p-1}u\\
      u(0,x) = u_{0}(x)
    \end{array}
  \right.
\end{equation}
has solution
\begin{equation*}
  u(t,x)= e^{i(t-t_{0})}\frac{\Delta}{2}u(t_{0}) - i\mu \int_{t_{0}}^{t}e^{i(t-s)\frac{\Delta}{2}}F(s)ds 
\end{equation*}
where $F(s) = |u(s,x)|^{p-1}u(s,x)$.

We will discuss classical solutions (LWP) on $[-T,T]$ or $[0,T]$. This means our
solutions are going to be continuous in time and space. Our initial data is in
$H_{x}^{s}(R^{d})$, $s\in\R$ where $s>d/2$. Sobolev embedding will give
continuity. The critcal threshold is $s_{c}=\frac{d}{2}-\frac{2}{p-1}$.


\begin{remark}
  Critical case means the time interval on which the solution is defined depends
  on the size of your initial data. If your data is small (i.e. is in a certain
  ball around $0$, then you have a unique solution. )
\end{remark}

The next theorem will use the fact that $H_{x}^{s}(R^{d})$ is an algebra for
certain values of $s$, depending on $d$.  This follows from the inequality
\begin{equation}\label{eq:Hs-algebra-inequality}
  \norm{fg}_{H^{s}_{x}} 
  \lesssim \norm{f}_{H^{s}_{x}}\norm{g}_{H_{x}^{s}}
\end{equation}
In the subcritical case, the inequality \cref{eq:Hs-algebra-inequality} will not
hold. We will use Strichartz estimates instead.

\begin{theorem}[Classical NLS solutions]
  \label{thm:classical-nls-solutions}
  Let $p>1$ be a prime number, $s\in \R$ such that $s>d/2$, and $\mu=\pm 1$.
  Then \cref{eq:nonlinear-schrodinger-equation} is unconditionally well-posed in $H_{x}^{s}(\R^d)$ in the
  subcritical sense: for all $R>0$ there exists $T=T(s,d,p,R)>0$ such that
  for all $u_{0}\in B_{R}=\left\{u_{0}\in H_{x}^{s}(\R^d),
    \norm{u_{0}}_{H_{x}^{s}(\R^d)}<R\right\}$ there exists a unique solution
  $u\in C_{t}^{0}H_{x}^{s}\left( \left[ -T,T \right]\times \R^d  \right)$ to
  \cref{eq:nonlinear-schrodinger-equation}. Furthermore, the map
  \begin{equation*}
    B_{R}\to C_{t}^{0}H_{x}^{s} \quad \text{given by}\quad u_{0}\mapsto u
  \end{equation*}
  is Lipshitz continuous.
\end{theorem}
\begin{proof}
  Fix $R>0$. We will decide what $T>0$ is later. Now we will apply the
  contraction mapping argument and will ultimately obtain unconditional
  well-posedness. Let
  \begin{equation*}
    \mathcal{S} = \mathcal{N} = C_{t}^{0}H_{x}^{s}
  \end{equation*}
  and consider the map $D:\mathcal{N} \to \mathcal{S}$ given by
  \begin{equation*}
    DF(t,x) = i\mu \int_{0}^{t}e^{i(t-s)\Delta/2}F(s,x)ds 
  \end{equation*}
  and let $N:\mathcal{S} \to \mathcal{N}$ be given by
  \begin{equation*}
    N(u(t,x)) = |u|^{p-1}u.
  \end{equation*}
  We will need to show that there exists a constant
  $C_{0}$ such that
  \begin{equation}\label{eq:14}
    \norm{DF}_{\mathcal{S}} \leq C_{0} \norm{F}_{\mathcal{N}}
  \end{equation}
  and 
  \begin{equation}\label{eq:15}
    \norm{Nu-Nv}_{\mathcal{N}} \leq  \frac{1}{2C_{0}}\norm{u-v}_{\mathcal{S}}
  \end{equation}
  (We are hiding the fact that we need $u_{\rm lin}\in B_{R/2}$. This won't
  screw anything up). First we will prove \cref{eq:14}:
  \begin{align*}
    \norm{DF}_{\mathcal{S}}
    &= \norm{\int_{0}^{t}e^{i(t-s)\Delta/2}F(s)ds }_{C_{t}^{0}H_{x}^{s}\left( [-T,T]\times \R^d \right)} \\
    &\leq C_{0}(t,d)\norm{F}_{H_{x}^{s}}.
  \end{align*}
  Next, to prove \cref{eq:15},
  \begin{align*}
    \norm{N(u)-N(v)}_{C_{t}^{0}H_{x}^{s}}
    &= \norm{|u|^{p-1}u-|v|^{p-1}v}_{C_{t}^{0}H_{x}^{s}}\\
    &\leq \norm{\left( |u|^{p-1}+|v|^{p-1} \right)\left(u-v\right)}_{C_{t}^{0}H_{x}^{s}}\\
    &\leq \tilde{C}_{0}(p,s,d,R)\norm{u-v}_{C_{t}^{0}H_{x}^{s}}.
  \end{align*}
  Now I want to apply the contraction mapping which means we have to have $T$
  sufficiently small (so that $\tilde{C}_{0}<\frac{1}{2C_{0}}$). Then
  contraction mapping implies that if if $u_{\rm lin }\in C_{t}^{0}H_{x}^{s}$,
  $\norm{u_{\rm lin}}_{C_{t}^{s}H_{x}^{s}}\lesssim R$, there exists a unique
  solution $u\in C_{t}^{\infty}H_{x}^{s}$ with
  $\norm{u}_{C_{t}^{0}H_{x}^{s}}<R$. And it also implies that the map
  $H_{x}^{s}\to C_{t}^{0}H_{x}^{s}$ given by $u \to u_{0}$ is Lipschitz on the
  ball in on radius of order $O(R)$. [This is proposition 1.38 in Terry Tao's
  notes]. This looks like \textbf{conditional} well-posedness. Next we explain
  why we also get unconditional well-posedness using a bootstrap argument. 

  We get uniqueness as long as $C_{t}^{0}H_{x}^{s}$ of the solution is $O(R)$.
  But $\norm{u_0}_{H_{x}^{s}}$ at most $R$ at time $t=0$. We prove unconditional
  uniqueness as follows. Let $u\in C_{t}^{0}H^{s}_{x}$ be what constructed
  earlier. So $\norm{u}_{\mathcal{S}}\leq C_{1}R$. Let $u^{*}\in
  C_{t}^{0}H^{s}_{x}$ is another solution. We'll show that $u^{*}$ is also in
  the ball. Let $H(t)$ be the proposition
  \begin{equation*}
    H(t):
    \norm{u^{*}}_{C_{t}^{0}H_{x}^{s}\left( [-t,t]\times R^{d} \right) } 
    \leq 2C_{1}R.
  \end{equation*}
  This is our bootstrap assumption. We will show that this implies that $u^{*}$
  actually lives in a smaller ball:
  \begin{equation*}
    C(t):
    \norm{u^{*}}_{C_{t}^{0}H_{x}^{s}\left( [-t,t]\times R^{d} \right) } 
    \leq C_{1}R,
  \end{equation*}
  which means we are in a smaller ball! It's like we want to be in the bathroom,
  and we are assuming that we are in the house (a decent assumption) and then
  finding out that we are actually in bathroom. The above bootstrap argument
  works for every $t$ up to some fixed maximal value, which is $T$. This implies
  unconditional LWP.   
\end{proof}
For the next theorem, we will need to introduce some new notation. If $s=0$ then $H^{s}=L^{2}$. 
\begin{equation*}
  s_{c}<0
\end{equation*}
and since $s_{c}=\frac{d}{2}-\frac{2}{p-1}$. Therefore
\begin{equation*}
  1<p <1+\frac{4}{d}
\end{equation*}
For the case with $s \leq d/2$ then sour solutions are in $H_{x}^{s}$ but this
does not give classical solutions because there is no algebra property for
$H^{s}$ when $s \leq d/2$.

\begin{theorem}[Subcritical $L^2$ Solutions to NLS]
  \label{thm:subcritical-L2-solutions-nls}  
\end{theorem}
\section{Lecture Notes:  2022-10-28}
Last time we proved LWP theory for classical $L^{2}$-subcritical oslution for
``our'' algebraic NLS.
\begin{equation}\label{eq:NLS-2022-10-28}
  \left\{
    \begin{array}{l@{\quad\quad}l} iu_{t}+\frac{1}{2}\Delta = \mu|u|^{p-1}u &
      p=\text{odd integer}\\
      u_{0}(x)= u(t_{0},x)& u:I\times \R^d \to \mathbb{C}
    \end{array}
  \right.
\end{equation}
and without loss of generality, one can assume that $t_{0}=0$.

Now, we want to show that we have less regular solutions, i.e. solutions in
$H_{x}^{s}$ for $s \leq \frac{d}{2}$. In this case, one does not have the
solutions in $L_{x}^{\infty}(\R^d)$ (recall $H^{s}\not\subset L^{\infty}$).

Now maybe we can no longer sray that for all time the solution is in
$L_{x}^{\infty}$, but maybe there is a time-averaged $L^{\infty}$ space such as
$L_{t}^{p-1}L_{x}^{\infty}(\R^d)$ such that the solutions with regularity less
than $\frac{d}{2}$ are contained by these spaces.

Heuristics suggest that hte existence of such time-averaged spaces is possible
because one does not expect ``energy'' of your solutions to be large on a large
time (because it is a dispersive PDE and one has Strichartz (linear)).

To this end we dfine $\mathcal{S}^{S}$ (Strichartz spaces) that capture all the
Strichartz norms at a certain regularity $H_{x}^{s}$ simultaneously.

Start with $L^{2}$ theory $(s=0,H^{0}=L^{2})$ which gives the Strichartz space
$\mathcal{S}^{0}=\mathcal{S}^{0}(I\times \R^d)$, $I\subset \R$, with norm defined as
\begin{equation*}
  \norm{u}_{\mathcal{S}^{0}(I\times \R^d)} := \sup_{(q,r) \text{ admissible}} \norm{u}_{L_{t}^{q}L_{x}^{r}}(I\times \R^d).
\end{equation*}
\pnote{What is the definition of admissibility here? Is it $2/q+d/r = d/2$? -m}
For $d=2$ we must be careful because the sets of admissible exponents is not
compact so the notion of a supremum needs to be adjusted.

\begin{remark}[Strichartz Space Norms]
  \label{rmk:strichartz-space-norms}
  We make the following observations:
  \begin{enumerate}
    \item The $\mathcal{S}^{0}(I\times \R^n)$ norm controls the
    $C_{t}^{0}L_{x}^{2}$ norm.
    \item $\mathcal{S}^{0}$ is a Banach space. Therefore we have a dual space,
    \begin{equation}\label{eq:13}
      \mathcal{N}^{0}(I\times \R^d):=[\mathcal{S}^{0}(I\times \R^d)]^{*}
    \end{equation}
    which is equipped with the norm
    \begin{equation*}
      \norm{F}_{\mathcal{N}^{0}(I\times \R^d)} 
      := \inf_{(q',r') \text{ admissible}} \norm{u}_{L_{t}^{q'}L^{r'}_{x}(I\times \R^d)}
    \end{equation*}
    \pnote{Should this be $Fu$ in the norm on the right-hand side?  -m}
    or equivalently,
    \begin{equation*}
      \norm{F}_{\mathcal{N}^{0}} \leq \norm{F}_{L_{t}^{q'}L_{x}^{r'}(I\times \R^d)}
    \end{equation*}
    provided that the right hand side is finite.
    \item $(\mathcal{S}^{1})^{*}\neq \mathcal{N}^{1}$
    \item Letting $F=i u_{t}+\frac{\Delta}{2}u$, we can compactly write the
    Strichartz estimates [(i),(ii)] \pnote{Not sure where this is referencing
      -m} in the following way
    \begin{equation}\label{eq:compactly-written-strichartz}
      \norm{u}_{\mathcal{S}^{0}(I\times \R^d)} 
      \lesssim d \norm{u(t_{0})}_{L_{x}^{2}(\R^d)}+\norm{F}_{\mathcal{N}^{0}(I\times \R^d)}
    \end{equation}
  \end{enumerate}
\end{remark}

\begin{theorem}[Subcritical LWP in Strichartz Space]
  \label{thm:subcritical-LWP-in-strichartz-space}
  Let $p$ be any $L^{2}$-subcritical exponent (i.e., $1<p<1+\frac{4}{d}$), and
  let $\mu=\pm 1$. Then \cref{eq:NLS-2022-10-28} is locally well-posed in
  $L_{x}^{2}(\R^d)$ in the subcritical sense. More specifically, for any $R>0$,
  there exists a constant $T=T(k,d,p,R)>0$ such that for all $u_{0}$ in
  $B_{R}:=\left\{u_{9}\in L_{x}^{2}(\R^d):
    \norm{u_{0}}_{L_{x}^{2}(\R^d)}<R\right\}$, it follows that there exists a
  unique solution $u$ to \cref{eq:NLS-2022-10-28} such that
  \begin{equation*}
    u\in \mathcal{S}^{0}\left( [-T,T]\times \R^d \right)
    \subset C_{t}^{0}L_{x}^{2}\left( [-T,T]\times \R^d \right),
  \end{equation*}
  and, furthermore, the map $B_{R}\to \mathcal{S}^{0}$ given by $u_{0}\mapsto u$
  is Lipshitz continuous.
\end{theorem}
\begin{proof}[Proof of \cref{thm:subcritical-LWP-in-strichartz-space}].
  The proof relies on the same \textbf{contraction mapping} argument that you find in Tao
  (prop 1.38), and have the same integral represntation (i.e. see
  \cref{eq:duhamel-general} of NLS via Duhamel's
  formula which leads to
  \begin{enumerate}
    \item $D:\mathcal{N} \to \mathcal{S}$ given by
    $DF(t):=-i \int_{0}^{t}e^{i(t-s)\frac{\Delta}{2}}F(s)ds $
    \item $N(u(t))=\mu |u|^{p-1}u$. 
  \end{enumerate}
  Here we use different choicese of $\mathcal{S}$ and $\mathcal{N}$ than in the
  classical solutions to the NLS proof. There if you recall we used
  $\mathcal{S}=\mathcal{N}=C_{t}^{0}H_{x}^{s}$.
  Here we take
  \begin{equation*}
    \mathcal{S} = \mathcal{S}^{0}(I\times \R^d)
    \quad \text{and} \quad
    \mathcal{N} = \mathcal{N}^{0}(I\times \R^d).
  \end{equation*}
  In order to place $u_{\rm lin}\in B_{\frac{\epsilon}{2}}$, we get from
  \cref{eq:compactly-written-strichartz} that we need to take $\epsilon=C_{1}R$,
  where $C_{1}=C_{1}(d)$ is some large constant depending on $d$.

  Now we need to return to proving the bound
  \begin{equation*}
    \norm{DF}_{\mathcal{S}^{0}} \leq C_{0} \norm{F}_{\mathcal{N}^{0}}.
  \end{equation*}
  This bound follow from \cref{eq:compactly-written-strichartz} (for some large
  $C_{0}$). Now we return to the estimate for $N$:
  \begin{equation*}
    \norm{N(u)-N(v)}_{\mathcal{N}^{p}} 
    \leq \frac{1}{2C_{0}} \norm{u-v}_{\mathcal{S}^{0}}.
  \end{equation*}
  Here $N(u) = |u|^{p-1}u$, so we want to show
  \begin{equation*}
    \norm{|u|^{p-1}u-|v|^{p-1}v}_{\mathcal{N}^{0}} \leq \frac{1}{2C_{0}} \norm{u-v}_{\mathcal{S}^{0}}
  \end{equation*}
  whenever we have $\norm{u}_{\mathcal{S}^{0}}\leq C,R$.

  Taking $(q,r)$ to be an admissible exponent pair, we have
  \begin{equation*}
    \frac{2}{q}+\frac{d}{r} = \frac{d}{2}
    \quad \text{and} \quad
    \frac{p}{r} = \frac{1}{r'}
  \end{equation*}
  The second equality refers to the nonlinearity in which we multiply $p$
  functions in $L^{r}$ and you land in $L^{r'}$. The hypotheses of the theorem
  are satisfied $(1 < p<1+\frac{4}{d})$, then $2<r<q<\infty$ and $(q,r)$ are
  admissible.

  Now, we can estimate the $\mathcal{N}^{0}$ norm by $L_{t}^{q'}L_{x}^{r'}$.
  Since $q>r$,  it follows that $\frac{p}{q}<\frac{1}{q'}$,  we can replace
  $L_{t}^{q'}$ by $L^{\frac{q}{p}}_{t}$ \pnote{(since
    $L^{\frac{q}{p}}_{t}\subset L^{q'}_{t}$)?}. This is the key point because by
  suing Holder inequality in time [i.e. since
  $\frac{1}{p}=\frac{1}{q}+\frac{1}{r}$, $r>0$, we have $\norm{f}_{L^{p}[-T,T]}\lesssim
  T^{1/r} \norm{f}_{L^{q}}$] and get a factor of $T^{\alpha}$ for some $\alpha>0$. 
\end{proof}

\end{document}

